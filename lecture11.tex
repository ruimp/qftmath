\chapter{Lecture 11 - Leander Stecker}

\section{Elliptic Operators and Complexes}

Let $M$ be a closed, oriented Riemannian manifold, and $E, F \to M$ complex vector bundles over $M$.
A \textbf{differential operator}
\begin{equation*}
  P \colon \Gamma (E) \longrightarrow \Gamma(F)
\end{equation*}
is a $\Cbb$-linear map that, in local coordinates $(U, x^i)$ of $M$ with local trivializations of $E$ and $F$, has the form
\begin{equation*}
  P = \sum_{|\alpha| \leq k} A^\alpha
  \frac{\partial^{|\alpha|} }{\partial x^{\alpha}} 
\end{equation*}
where $\alpha$ is a multi-index and
\begin{equation*}
  A^\alpha \in \Hom ( E |_U, F |_U )
\end{equation*}
is a bundle map over $U$.
The number $k \in \Zbb_{\geq 0}$ is called the \textbf{order} of $P$.

\begin{definition}
  The \textbf{principal symbol} $\sigma(P)$ is a section of the pullback bundle
  $\pi^* \Hom(E, F)$ over $T^*M$, as in the following diagram.
  \begin{equation*}
    \begin{tikzcd}[sep=large]
      |[label={-45:\lrcorner}]|
      \pi^* \Hom(E, F) \arrow[r] \arrow[d] &
      \Hom(E, F) \arrow[d] \\
      T^* M \arrow[r, "\pi"] &
      M
    \end{tikzcd}
  \end{equation*}
\end{definition}

At $\xi \in T^* M$, $\sigma(P)$ is given by
\begin{equation*}
  \sigma(P)_\xi = \sum_{|\alpha| \leq k} A^\alpha (x) \xi_\alpha
\end{equation*}
where for $\alpha = (\alpha_1, \dots, \alpha_n)$, we write $\xi = \xi_i \drm x^i$.
Since $\sigma(P)$ is the only symbol we will need, let us just call it the \textit{symbol} of $P$.

\begin{lemma}
  The symbol of $P$ is equivalently defined by
  \begin{equation*}
    \sigma(P)_\xi = \lim_{t \to \infty} (it)^{-k} \euler^{-i t f}
    P \euler^{i t f}
  \end{equation*}
  where $f$ is any smooth function such that
  \begin{equation*}
    \drm f (x) = \xi.
  \end{equation*}
\end{lemma}
\begin{proof}
  In a local trivialization over a chart, for a local section $s$, we compute
  \begin{align*}
    P \euler^{i t f} s
    &= \sum_{|\alpha| \leq k} A^\alpha
    \frac{\partial^{|\alpha|}}{\partial x^\alpha}
    \bigl( \euler^{i t f} s \bigr) \\
    &= \sum_{|\alpha| \leq k} (i t)^k
    \frac{\partial^{|\alpha|} f}{\partial x^\alpha} \euler^{i t f}
    A^\alpha s +
    \substack{\text{lower} \\ \text{order in } t}
  \end{align*}
  hence
  \begin{equation*}
    \lim_{t \to \infty} (it)^{-k} \euler^{-i t f} P \euler^{i t f} s
    = \sum_{|\alpha| = k} \drm f \biggl( \frac{\partial^|\alpha|}{\partial x^\alpha} \biggr) A^\alpha s.
  \end{equation*}
  Evaluating at $x$ gives $\sigma(P)_\xi$.
\end{proof}

\begin{example}
  The exterior derivative
  \begin{equation*}
    \drm \colon \Omega^\bullet(M) \longrightarrow \Omega^\bullet (M)
  \end{equation*}
  can be writen in local coordinates as
  \begin{equation*}
    \drm = \sum_{i=1}^n \frac{\partial}{\partial x^i} \drm x^i \wedge
  \end{equation*}
  therefore
  \begin{equation*}
    \sigma(\drm)_\xi = \sum_{i=1}^n \xi_i \drm x^i \wedge = \xi \wedge
  \end{equation*}
  so the symbol is given by the wedge product.
\end{example}

\begin{example}
  The Laplacian
  \begin{equation*}
    \laplace \colon \Crm^\infty(M) \longrightarrow \Crm^\infty (M)
  \end{equation*}
  can be writen in normal coordinates around $p \in M$
  \begin{equation*}
    \laplace |_p = - \sum_{i = 1}^n
    \biggl( \frac{\partial}{\partial x^i} \biggr)^2
  \end{equation*}
  hence $\sigma (\laplace)_\xi = - |\xi|^2$.
\end{example}

\begin{example}
  The Hodge Laplacian
  \begin{equation*}
    \laplace \colon \Omega^\bullet (M) \longrightarrow \Omega^\bullet (M)
  \end{equation*}
  is given by
  \begin{align*}
    \laplace 
    &= \drm \drm^\hodge + \drm^\hodge \drm \\
    &= - \sum_{i=1}^n \biggl( \frac{\partial}{\partial x^i} \biggr)^2
    \Bigl( \iota_{\frac{\partial}{\partial x^i}} \drm x^i
    + \drm x^i \iota_{\frac{\partial}{\partial x^i} }\Bigr) \\
    &= - \sum_{i=1}^n
    \Bigl[ \iota_{\frac{\partial}{\partial x^i}}, \drm x^i \Bigr]
    \biggl( \frac{\partial}{\partial x^i} \biggr)^2
  \end{align*}
  where $\iota_{\frac{\partial}{\partial x^i}}$ denotes the interior product and $\drm x^i$ is the operator $\drm x^i \wedge$ in $\Omega^\bullet (M)$.

  Remarkably, we still have $\sigma(\laplace)_\xi = - |\xi|^2$. To see this, we need the following proposition.
  
  \begin{proposition}
    For differential operators $P$ and $P'$ we have
    \begin{equation*}
      \sigma(P)_\xi \sigma(P')_\xi = \sigma(P P')_\xi.
    \end{equation*}
  \end{proposition}
  \begin{proof}
    If $P$ has order $k$ and $P'$ has order $k'$, then $P P'$ has order $k + k'$ and
    \begin{align*}
      &\quad \lim_{t \to \infty}
      \Bigl( (i t)^{-k} \euler^{-i t f} P \euler^{i t f} \Bigr)
      \Bigl( (i t)^{-k'} \euler^{- i t f} P' \euler^{i t f} \Bigr) \\
      &= \lim_{t \to \infty}
      \Bigl( (i t )^{-k-k'} \euler^{- i t f} P P' \euler^{i t f} \Bigr).
    \end{align*}
  \end{proof}

  We also need that
  \begin{equation*}
    \sigma( \drm^\hodge)_\xi = - \iota_{\xi^\#}
  \end{equation*}
  where $\xi^\# \in T_x M$ is the unique vector such that $\xi = g(\xi^\#, \cdot)$. Then
  \begin{align*}
    \sigma(\laplace)_\xi
    &= \sigma(\drm \drm^\hodge + \drm^\hodge \drm)_\xi \\
    &= \sigma(\drm \drm^\hodge)_\xi + \sigma (\drm^\hodge \drm)_\xi \\
    &= \bigl[\xi, - \iota_{\xi^\#}\bigr] \\
    &= - | \xi |^2.
  \end{align*}
\end{example}

\begin{definition}
  An operator $P \colon \Lambda (E) \to \Lambda(F)$ is \textbf{elliptic} if the symbol $\sigma(P)_\xi$ is invertible for all $\xi \neq 0$.
\end{definition}
Remember that, for all $\xi \in T_x^*M$, the symbol is a linear operator $\sigma(P)_\xi \colon E_x \to F_x$.

\begin{definition}
  An operator $P : \Lambda(E) \to \Lambda(E)$ is called a \textbf{generalized Laplacian} if
  \begin{equation*}
    \sigma(P)_\xi = - |\xi|_{\tilde{g}}^2
  \end{equation*}
  for some metric $\tilde{g}$ on $M$, not necessarily the one we started with.
\end{definition}
The previous definition comes from \cite{BGVHeat96}, but we have taken the opposite sign convention.

\section{Elliptic Complexes}

Suppose $P \colon \Gamma(E) \to \Gamma(F)$ is elliptic. The symbol $\sigma(P)$ defines a bundle map over $T^*M$
\begin{equation*}
  \begin{tikzcd}
    0 \arrow[r] &
    \pi^* E \arrow[r, "\sigma(P)"] &
    \pi^* F \arrow[r] & 0
  \end{tikzcd}
\end{equation*}
which is exact over $T^*M$ away from the zero section.
Now let $(E^\bullet, Q)$ be the $\Zbb$-graded vector bundle $E^\bullet \to M$, and
\begin{equation*}
  Q \colon \Gamma(E^\bullet) \longrightarrow \Gamma(E^\bullet)
\end{equation*}
a differential operator of cohomological degree $1$ such that $Q^2 = 0$.

\begin{definition}
  The complex $(E^\bullet, Q)$ is an \textbf{elliptic complex} if the symbol complex $(\pi^* E, \sigma(Q))$ is exact over $T^*M \to M$.
\end{definition}

\begin{example}
  To the de Rham complex
  \begin{equation*}
    \begin{tikzcd}
      0 \arrow[r] &
      \Omega^0 (M) \arrow[r, "\drm"] &
      \Omega^1 (M) \arrow[r, "\drm"] &
      \dots \arrow[r, "\drm"] &
      \Omega^n (M)
    \end{tikzcd}
  \end{equation*}
  corresponds the symbol complex for $\xi \in \Omega^1 (M)$
  \begin{equation*}
    \begin{tikzcd}
      0 \arrow[r] &
      \Omega^0 (M) \arrow[r, "\xi"] &
      \Omega^1 (M) \arrow[r, "\xi"] &
      \dots \arrow[r, "\xi"] &
      \Omega^n (M)
    \end{tikzcd}.
  \end{equation*}
  If $\xi \neq 0$ and $\xi \wedge \beta = 0$, then $\beta = \xi \wedge \alpha$ for some $\alpha \in \Omega^\bullet(M)$. We conclude that the symbol complex is exact, and thus $(\Omega^\bullet, \drm)$ is an elliptic complex.
\end{example}

\begin{example}
  The Yang-Mills complex is not elliptic. We have
  \begin{equation*}
    \begin{tikzcd}
      0 \arrow[r] &
      \Omega^0(M, \gfrak) \arrow[r, "\drm"] &
      \Omega^1(M, \gfrak) \arrow[r, "\drm \hodge \drm"] &
      \Omega^{n-1}(M, \gfrak) \arrow[r, "\drm"] &
      \Omega^n (M, \gfrak)
    \end{tikzcd}
  \end{equation*}
  with corresponding symbol complex
  \begin{equation*}
    \begin{tikzcd}
      M \times \gfrak \arrow[r, "\xi \wedge"] &
      T^* M \otimes \gfrak \arrow[r, "Q"] &
      \bigwedge^{n-1} T^* M \otimes \gfrak \arrow[r, "\xi \wedge"] &
      \bigwedge^n T^* M \otimes \gfrak
    \end{tikzcd}
  \end{equation*}
  where
  \begin{equation*}
    Q = \sigma(\drm \hodge \drm)_\xi \alpha \\
      = \sum_{j, k} \xi_j \xi_k \drm x^i \wedge
      \hodge (\drm x^i \wedge \alpha).
  \end{equation*}
  For $n \geq 2$ this complex is not exact, hence the original complex not elliptic.
\end{example}
