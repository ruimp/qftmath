\chapter{Lecture 20240429}

(John Huerta)

\section{Ultimate Goal}

Define and use the Feynman ``path'' integral
\begin{equation*}
  \int_{\phi \in \Fcal} e^{-\frac{S (\phi)}{\hbar}}
\end{equation*}
(Euclidean field theory)

In the constructive track: see Gonçalo on how to do this. In the BV
track: we will produce a formal power series in $\hbar$.

\section{Recall}

\begin{itemize}
\item From now on: We work {\em perturbatively}, i.e., {\em formally}
  (in Algebraic Geometry speak), i.e., in {\em formal power series},
  i.e., {\em infinitesimally}.
\item Now $M$ is going to be a finite dimensional manifold, denoting space-time. E.g.,
  \begin{align*}
    M&=\Rbb^d \intertext{or} \\
    M&=\mathrm{pt}
  \end{align*}
\item $\Fcal$ always denotes the naive fields, a sheaf of vector
  spaces on $M$; specifically, sections of some vector bundle
  $F\longrightarrow M$.

  Example: Yang-Mills fields for a trivial $G$-bundle
  $M\times G \longrightarrow M$, then
  $\Fcal (M)=\Omega^1 (M,\gfrak)$, where
  $\gfrak=\mathrm{Lie} (G)$.
\item $\Ecal$ (``extended''), the space of BV-fields, a sheaf of {\em
    graded} vector spaces over $M$, sections of a graded vector bundle
  $E\longrightarrow M$. $\Ecal^0 (M)=\Fcal (M)$.

  In the Yang-Mills example, where $d=\mathrm{dim} M$
  \begin{equation*}
    \Ecal(M) = \underbrace{\Omega^0 (M,\gfrak)}_{\text{``ghosts''}}^{-1}
    \oplus\underbrace{\Omega^1(M,\gfrak)}_{\text{``fields''}}^{0}
    \oplus\underbrace{\Omega^{d-1} (M,\gfrak)}_{\text{``anti-fields''}}^{1}
    \oplus\underbrace{\Omega^d(M,\gfrak)}_{\text{``anti-ghosts''}}^{2}
  \end{equation*}
\end{itemize}

\section{BV Formulation of Gauge Theory}

Input: Naive gauge theory
\begin{equation*}
\underbrace{\Lcal}_{
  \substack{\text{Lie algebra of}\\
    \text{infinitesimal gauge}\\
    \text{transformations}} } \curvearrowright
\underbrace{\Fcal}_{\text{space of naive fields}}\,.
\end{equation*}
The action may be non-linear. In the Young-Mills example it is an
affine action
\begin{equation*}
  \Omega^0 (M\gfrak)\curvearrowright\Omega^1 (M\gfrak)\,.
\end{equation*}
There is a two-step process to writing down the gauge theory:
\begin{enumerate}
\item Take the {\em ``stacky quotient''}
  \begin{equation*}
    \Fcal \rightsquigarrow \Fcal \hquot \Lcal \qquad \text{(this lecture)}
  \end{equation*}
\item Take the derived critical locus of $S_\text{gauge}$:
  \begin{equation*}
    T^*[-1]\left(\Fcal\hquot\Lcal\right)\,. \qquad \text{(already done)}
  \end{equation*}
\end{enumerate}

\section{Lightning Fast Introduction to Derived Invariants}
$\gfrak$ a finite dimensional Lie algebra, $R$ a finite dimensional
representation of $\gfrak$
\begin{equation*}
  \gfrak\rightarrow\mathfrak{gl} (R)
\end{equation*}
over some field $k\in\{\Rbb,\Cbb\}$.
Observe that
\begin{align*}
  R^{\gfrak}&=\{v\in \Rbb \mid Xv=0 \text{ for all } X\in\gfrak \}\\
          &=\mathrm{Hom}_{\gfrak} (k,R)
\end{align*}
Derived version $\mathrm{Hom} \rightsquigarrow \Rbb\mathrm{Hom}$.

Try $R^{\mathrm{h}\gfrak}=\Rbb\mathrm{Hom}_{U}\gfrak (k,R)$, where
$U$ is the eveloping algebra. I.e.,
\begin{equation*}
  U\gfrak=\frac{T\gfrak}{x\otimes{}y-y\otimes{}x-[x,y]}
\end{equation*}
where $Tgfrak$ is the tensor algebra.
\begin{fact}
  $\mathrm{Rep_{\gfrak}}\simeq U\gfrak\mathrm{-mod}\,.$
\end{fact}
To compute $\Rbb\mathrm{Hom}_{U\gfrak} (k,R)$ we need to resolve $k$
or $R$ as $U\gfrak$ modules.

Similar to the Koszul complex

TODO
\section{Back to Yang-Mills}

TODO