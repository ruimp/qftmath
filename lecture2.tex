\chapter{Lecture 2}

In this lecture we consider a naive example illustrating how the Euler-Lagrange equations lead us to classical BV theories.

\begin{example}
  Let $\Fcal$ be a finite-dimensional vector space encoding the naive space of fields and consider an action
  \begin{equation*}
    S \colon \Fcal \longrightarrow \Rbb.
  \end{equation*}
  We say that $S$ is a naive action because adding additional terms might be necessary to guarantee that it satisfies the CME.
  The solutions to the Euler-Lagrange equation are fields $f \in \Fcal$ such that ${\drm S} (f) = 0$.

  In general, if $\Fcal = M$ for some finite-dimensional manifold $M$, we say that critical points of the action form the \textbf{critical locus} of $S$
  \begin{equation*}
    \Crit (S) = \bigl\{ p \in M \bigm| \drm S (p) = 0 \bigr\}.
  \end{equation*}
  Alternatively, we can characterize the critical locus of $S$ as an intersection in $T^* M$
  \begin{equation*}
    \Crit (S) = \Graph( \drm S) \cap \Graph (M)
  \end{equation*}
  where we identify $M$ with the zero section. It follows that
  \begin{equation*}
    \Oscr( \Crit (S) ) =
    \Oscr( \Graph (\drm S ) ) \otimes_{\Oscr(T^* M)} \Oscr(M).
  \end{equation*}
  We are going to consider a derived version of this construction, where the tensor product $\otimes$ is replaced by a derived tensor product $\otimes^{\Lbb}$. This raises the obvious questions:
  \begin{figure}[ht]
    \includestandalone{bad_intersections}
    \centering
    \caption{Well-behaved (in green) and badly-behaved (in red) intersection points.}
    \label{fig:bad_intersections}
  \end{figure}
  \begin{itemize}
    \item \textbf{Why?} The intersection might not be well-behaved, in the sense that $\drm S$ and the zero section might not intersect transversally, or even smoothly, at every point, as illustrated in figure \ref{fig:bad_intersections}. The derived approach allows us to study these badly-behaved points using Serre's intersection formula.
    \item \textbf{How?} We replace $\Oscr (\Graph(\drm S)) \otimes_{\Oscr (T^*M)} \Oscr(M)$ with a dg commutative algebra $A$ such that
    \begin{equation*}
      \Hrm^0 A = \Oscr (\Graph(\drm S)) \otimes_{\Oscr (T^*M)} \Oscr(M).
    \end{equation*}
    To compute the derived tensor product $\otimes^{\Lbb}$ we need to resolve either $\Oscr(M)$ or $\Oscr(\Graph(\drm S))$ in $\Oscr(T^* M)$-modules. Let us make use of Darboux coordinates to write
    \begin{align*}
      \Oscr(\Graph(\drm S)) &=
      \bigslant{\Oscr(T^* M)}{ \bigl( f \vert_{\Graph(\drm S)} = 0 \bigr) } \\
                            &= \bigslant{ \Oscr( T^* M )}{ \bigl( p_{\mu} - \partial_\mu S \bigr) }.
    \end{align*}
    Consider the resolution
    \begin{equation*}
      \begin{tikzcd}[sep=large]
        \dots \arrow[r] &
        \Oscr(T^* M) (\xi_1, \dots, \xi_n) \arrow[r, "\xi_{\mu} \mapsto p_{\mu} - \partial_{\mu} S"] &
        \Oscr(T^* M) \arrow[d] \\ &&
        \Oscr(\Graph(\drm S))
      \end{tikzcd}
    \end{equation*}
    which we extend to the left as a Koszul complex
    $K^{-p} = \bigwedge_{\Oscr(T^* M)}^p (\xi_1, \dots, \xi_n)$
    with differential 
    \begin{equation*}
      \drm = \sum_{\mu} (p_{\mu} - \partial_{\mu} S) \frac{\partial}{\partial \xi_{\mu}}.
    \end{equation*}
    This complex freely resolves $\Oscr(\Graph(\drm S))$. Alternatively, $(K^\bullet, \drm)$ admits a coordinate free description where
    \begin{equation*}
      K^{-p} = \Oscr(T^* M) \otimes_{\Oscr(M)} \Xfrak^p (M).
    \end{equation*}
    A model for $\Oscr(\Graph(\drm S)) \otimes_{\Oscr(T^* M)}^{\Lbb} \Oscr(M)$ is given by
    \begin{equation*}
      \Oscr(\dCrit(S)) = K^{-\bullet} \otimes_{T^* M} \Oscr(M)
    \end{equation*}
    which we call the \textbf{derived critical locus}.
    Now note that
    \begin{equation*}
      \Oscr(T^* M) \otimes_{\Oscr(M)} \PV^\bullet(M) \otimes_{\Oscr(T^* M)} \Oscr(M) \cong \PV^\bullet(M)
    \end{equation*}
    where $\PV^\bullet(M)$ denotes the complex of polyvector fields on M. The differential is given by contracting with $\drm S$ so we write
    \begin{equation*}
      \Oscr(\dCrit(S)) = (\PV^\bullet(M), -\iota_{\drm S}).
    \end{equation*}
  \end{itemize}
\end{example}
