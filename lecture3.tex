\chapter{Lecture 3}

We want to sketch how to go from the Yang-Mills action
\begin{equation*}
  S^{\text{naive}} (A) = \int_{M^n} \tr (F_A \wedge \star F_A)
\end{equation*}
to the Yang-Mills classical BV theory
\begin{equation*}
  \begin{tikzcd}
    \underbrace{\Omega^0 (M, \gfrak)}_{\text{ghosts}}^{-1}
    \arrow[r, "\drm"] &
    \underbrace{\Omega^1 (M, \gfrak)}_{\text{fields}}^{0}
    \arrow[r, "\drm \star \drm"] &
    \underbrace{\Omega^{n-1} (M, \gfrak)}_{\text{antifields}}^{1}
    \arrow[r, "\drm"] &
    \underbrace{\Omega^n (M, \gfrak)}_{\text{antighosts}}^{2}
  \end{tikzcd}
\end{equation*}
with BV action
\begin{equation*}
  S^{\text{BV}}(e) = \langle e, Q e \rangle + I(e)
\end{equation*}
where
\begin{equation*}
  \langle e, f \rangle = \int_{M^n} \tr (e \wedge f)
\end{equation*}
is the $-1$-shifted symplectic structure. There are some points to motivate:
\begin{enumerate}[i)]
  \item \textbf{fields} $\rightsquigarrow$ \textbf{fields and antifields:} coming from the derived critical locus $\text{dCrit}(S)$;
  \item \textbf{ghosts:} coming from taking the derived coinvariants of $\gfrak \curvearrowright V$.
\end{enumerate}

For Yang-Mills spacetime is a manifold $M^n$ and $\Omega^1(M, \gfrak)$ is the space of fields.
However, in what follows, let $M$ be the space of fields.
Recall that
\begin{align*}
  \Crit(S) &= \bigl\{ p \in M \bigm| \drm S_p = 0 \bigr\} \\
           &= \Graph (\drm S) \cap M
\end{align*}
in $T^* M$. Dually
\begin{equation*}
  \Oscr \bigl( \Crit(S) \bigr)
  = \Oscr \bigl( \Graph (\drm S) \bigr) \otimes_{\Oscr(T^* M)} \Oscr(M).
\end{equation*}
By homological yoga, taking the derived intersection means that we replace the tensor product $\otimes$ with the derived tensor product $\otimes^{\Lbb}$.
To find $\dCrit (S)$ we resolve either $\Oscr \bigl( \Graph (\drm S) \bigr)$ or $\Oscr(M)$ as $\Oscr(T^* M)$-modules. 
Last time, we wrote the Koszul complex
\begin{equation*}
  K^{-p} = \PV^p (M) \otimes_{\Oscr(M)} \Oscr (T^* M)
\end{equation*}
where $\PV^p = \bigwedge^p \Xfrak(M)$ and differential
\begin{equation*}
  Q : v_1 \wedge \dots \wedge v_k \otimes 1 \longmapsto
  \sum_{i=1}^k (-1)^{i+1} v_1 \wedge \dots \wedge \hat{v}_i \wedge \dots \wedge v_k
  \otimes \bigl( p(v_i) - \drm S(v_i) \bigr)
\end{equation*}

\begin{exercise}
  Check that $\Hrm^0 (K^{\bullet}, Q) \cong \Oscr \bigl( \Graph (\drm S) \bigr)$, so
  \begin{equation*}
    \dCrit (S) \cong K^{\bullet} \otimes_{\Oscr(T^* M)} \Oscr (M) \cong \PV^{\bullet} (M)
  \end{equation*}
  and thus $\Oscr \bigl( \dCrit(S) \bigr) \simeq (\PV^\bullet, - \iota_{\drm S})$.
\end{exercise}

\begin{exercise}
   Show that $\Hrm^0 \Oscr( \dCrit ) \cong \Oscr (\Crit)$.
\end{exercise}

We can enhance $\Oscr \bigl( \dCrit (S) \bigr)$ to a sheaf on $M$. Following Grothendieck
\begin{equation*}
  \dCrit(S) = \bigl( M, \PV_M^\bullet, - \iota_{\drm S} \bigr)
\end{equation*}
is an example of a \textbf{dg manifold}.

\begin{definition}
  A dg manifold is a smooth manifold $M$ with a sheaf $\Oscr_M$ of \textbf{dg commutative algebras} (DGCAs) locally isomorphic to $\Oscr_M(U) \cong \bigwedge \Ecal (U)$, where $\Ecal$ are the smooth sections of $E \rightarrow M$.
\end{definition}

Ignoring the differential, we get a sheaf $\bigl(M, \PV_M^\bullet\bigr)$ on $M$ such that
\begin{equation*}
  \PV_M = \bigwedge \Xfrak_M \cong \Sym \Xfrak[1].
\end{equation*}
The underlying graded manifold of $\dCrit(S)$ is
\begin{equation*}
  T^*[-1]M = \bigl( M, \Sym \Xfrak[1] \bigr)
\end{equation*}
displaying the following properties:
\begin{enumerate}[i)]
  \item the graded manifold $T^*[-1]M$ is a $-1$-shifted symplectic graded manifold just as $T^*M$ is a $0$-shifted symplectic manifold;
  \item Induced from the $-1$-shifted symplectic structure we get a $1$-shifted Poisson bracket on $\Oscr \bigl( T^* [-1] M \bigr) = \PV(M)$ known as the \textbf{Schouten bracket}
  \begin{align*}
    \{ f, g \} &= 0, \\
    \{v, f \} &= v f, \\
    \{ v, w \} &= [v, w], \\
    \{u, v \wedge w \} &= \{u, v\} \wedge w + v \wedge \{ u, w \}
  \end{align*}
for $f, g \in \Oscr(M)$ and $u, v, w \in \PV^{-1}(M)$.
\end{enumerate}

\begin{exercise}
  Show that $-\iota_{\drm S} = \{ S, \cdot \}$.
\end{exercise}

\begin{definition}
  A $\Pbb_0$ algebra $\bigl( A, \drm, \{ \cdot, \cdot \} \bigr)$ is a DGCA $(A, \drm)$ equipped with a $1$-shifted Poisson bracket $\{ \cdot, \cdot \}: A \otimes A \rightarrow A$ obeying:
  \begin{enumerate}[i)]
    \item \textbf{graded skew-symmetry:}
      \begin{equation*}
        \{x, y\} = - (-1)^{(|x|+1)(|y|+1)} \{y, x\};
      \end{equation*}
    \item \textbf{graded Poisson identity}:
      \begin{equation*}
        \{x, yz \} = \{x, y\} z + (-1)^{(|x|+1)|y|} y \{x, z\}
      \end{equation*}
      so $\{x, \cdot \}$ is a degree $|x|+1$ derivation;
    \item \textbf{graded Jacobi identity}:
      \begin{equation*}
        \{x, \{y, z\} \} = \{ \{x, y\}, z \} + (-1)^{(|x|+1)(|y|+1)} \{y, \{x, z\} \};
      \end{equation*}
    \item \textbf{compatibility with differential:}
      \begin{equation*}
        \drm \{x, y\} = \{ \drm x, y \} + (-1)^{|x|+1} \{x, \drm y \}.
      \end{equation*}
  \end{enumerate}
\end{definition}

\begin{exercise}
  Check that the Schouten bracket defines a $\Pbb_0$-algebra on $\Oscr\bigl(T^*[-1] M \bigr)$.
\end{exercise}
