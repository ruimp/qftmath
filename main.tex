%! TeX program = lualatex
\documentclass[a4paper, 12pt, oneside]{memoir}

\setulmarginsandblock{3cm}{3cm}{*}
\setlrmarginsandblock{3cm}{3cm}{*}
\checkandfixthelayout

% Style
\chapterstyle{dash}
\renewcommand*{\chaptitlefont}{\normalfont\Huge\bfseries\sffamily}
\renewcommand*{\chapnamefont}{\normalfont\huge\bfseries\sffamily}
\renewcommand*{\chapnumfont}{\normalfont\huge\bfseries\sffamily}
\setsecheadstyle{\normalfont\Large\bfseries\sffamily}
%\renewcommand{\chaptername}{Lecture}

% Tikz
\usepackage{tikz}
\usetikzlibrary{cd}

% Bibliography
\usepackage{hyperref}
\usepackage[backend=biber, style=alphabetic]{biblatex}
\addbibresource{library.bib}


% Tables
\usepackage{booktabs}
\usepackage{makecell}

% Typesetting
\usepackage[english]{babel}
\usepackage{csquotes}
\usepackage{libertinus-otf}
%\OnehalfSpacing

% Theorem environments
\usepackage{amsthm}
\usepackage{thmtools}
\declaretheorem[style=definition]{definition}
\declaretheorem[style=definition, numbered=no]{example}
\declaretheorem[style=definition, numbered=no]{exercise}
\declaretheorem[style=definition, numbered=no]{fact}


% Symbols
\usepackage{slashed}

% Macros
\newcommand{\Ecal}{\mathcal{E}}
\newcommand{\Fcal}{\mathcal{F}}
\newcommand{\Lcal}{\mathcal{L}}
\newcommand{\Ocal}{\mathcal{O}}
\newcommand{\Xfrak}{\mathfrak{X}}
\newcommand{\gfrak}{\mathfrak{g}}
\newcommand{\Rbb}{\mathbb{R}}
\newcommand{\Cbb}{\mathbb{C}}
\newcommand{\Lbb}{\mathbb{L}}
\newcommand{\Pbb}{\mathbb{P}}
\newcommand{\kbold}{\pmb{k}}
\newcommand{\drm}{\mathrm{d}}
\newcommand{\Hrm}{\mathrm{H}}
\newcommand{\dd}{\, \mathrm{d}}
\newcommand{\hodge}{{\star}}
\newcommand{\bigslant}[2]{{\left.\raisebox{.2em}{$#1$}\middle/\raisebox{-.2em}{$#2$}\right.}}
\DeclareMathOperator{\tr}{tr}
\DeclareMathOperator{\Crit}{Crit}
\DeclareMathOperator{\dCrit}{dCrit}
\DeclareMathOperator{\Graph}{Graph}
\DeclareMathOperator{\PV}{PV}
\DeclareMathOperator{\Sym}{Sym}
\DeclareMathOperator{\hquot}{/\mkern-6mu/}

\title{Quantum Field Theory for Mathematicians}
\author{}
\date{}

\begin{document}
\maketitle

\chapter{Lecture 1}

\section{History}

A brief history of the development of the Batallin-Vilkovisky (BV) formalism:

\begin{center}
  \begin{tabular}{ccccc}
    Date & People & What & Why & Techniques \\
    \midrule
    1969 &
    \makecell{ Faddeev \\ and Popov } &
    \makecell{ Gauge fixing \\ (adding ghosts) } &
    \makecell{ Quantize \\ Yang-Mills } &
    \makecell{ Berezinian \\ integration } \\
    \addlinespace
    1973 & 
    \makecell{ 't Hooft and \\ Veltman } &
    \makecell{ Quantized \\ Yang-Mills } &
    \makecell{ Quantize \\ Yang-Mills } &
    \makecell{ Feynman \\ diagrams } \\
    \addlinespace
    1975 &
    \makecell{ Becchi, Rouet, \\ Stora, Tyutin \\ (BRST) } &
    \makecell{ Cohomological \\ theory to quantize \\ Yang-Mills } &
    \makecell{ Understanding \\ 't Hooft \\ and Veltman } &
    \makecell{ Derived invariants \\ (Lie algebra \\ cohomology) } \\
    \addlinespace
    1981 &
    \makecell{ Batallin and \\ Vilkovisky \\ (BV) } &
    \makecell{ Quantize systems \\ with complicated \\ gauge symmetries } &
    Supergravity &
    \makecell{ Derived \\ intersections \\ (Koszul complexes) } \\
    \addlinespace
    1992 & Henneaux &
    \makecell{ Quantize \\ Yang-Mills \\ using BV } &
    \makecell{ Analyze \\ Yang-Mills \\ using BV } &
    \makecell{ Derived \\ intersections} \\
    \addlinespace
    2007 & Costello &
    \makecell{ Combine BV \\ with effective \\ field theory } &
    \makecell{ Make BV \\ quantization \\ rigorous } &
    \makecell{ Derived everything, \\ analysis, and \\ homotopy theory }
  \end{tabular}
\end{center}

% ╭──────────────────────────────────────────────────────────╮
% │                        References                        │
% ╰──────────────────────────────────────────────────────────╯
\section{References}

The main references will be:

\begin{itemize}
  \item Costello - Renormalization and Effective Field Theory \cite{CosRenormalization11};
  \item Elliot, Williams, Yoo - Asymptotic Freedom in the BV Formalism \cite{EWYAsymptotic18};
  \item Gwilliam - Factorization algebras and free field theories \cite{GwiFactorization}.
\end{itemize}

% ╭──────────────────────────────────────────────────────────╮
% │                Roadmap to BV quantization                │
% ╰──────────────────────────────────────────────────────────╯
\section{Roadmap to BV quantization}

\begin{figure}
  \includestandalone{roadmap_to_BV}
  \caption{Roadmap to BV quantization.}
\end{figure}

The space of fields $\Ecal^\bullet$ is a cochain complex
\begin{equation*}
  \begin{tikzcd}
    \dots \arrow[r] &
    \Ecal^{-1} \arrow[r, "Q"] &
    \Ecal^0 \arrow[r, "Q"] &
    \Ecal^1 \arrow[r, "Q"] &
    \Ecal^2 \arrow[r] &
    \dots
  \end{tikzcd}
\end{equation*}
equipped with a differential $Q$ such that $Q^2 = 0$. Moreover, $\Ecal$ admits a $-1$-shifted symplectic structure, that is, there exists a nondegenerate pairing of degree $-1$
\begin{equation*}
  \langle \cdot , \cdot \rangle \colon
  \Ecal \otimes \Ecal \longrightarrow \Rbb [-1]
\end{equation*}
such that $\langle x, y \rangle = -(-1)^{(|x|+1)(|y|+1)} \langle y, x \rangle$.
This structure defines a $+1$-shifted Poisson bracket
\begin{equation*}
  \{ \cdot, \cdot \} \colon
  \Oscr( \Ecal ) \otimes \Oscr( \Ecal ) \longrightarrow \Oscr( \Ecal )
\end{equation*}
where $\Oscr ( \Ecal ) \cong \mathrm{Sym}^\bullet (\Ecal^\vee)$ is the (graded) commutative algebra of polynomial functions on the dual complex $\Ecal^\vee$.
Consider a function $S \in \Oscr (\Ecal)$ obeying the \textbf{classical master equation} (CME)
\begin{equation*}
  \{ S, S \} = 0.
\end{equation*}
The data $(\Ecal^\bullet, \langle \cdot, \cdot \rangle, S)$ defines a \textbf{classical BV theory}.
The CME implies that $\{ S, \cdot \}$ is a differential making $(\Oscr ( \Ecal ), \{S, \cdot \} )$ into a cochain complex such that
\begin{equation*}
  \Hrm^0 \Oscr ( \Ecal ) \cong
  \Oscr( \mathrm{Crit} (S) ),
\end{equation*}
where $\mathrm{Crit} (S)$ denotes the critical locus of $S$.
We will restrict to $S$ of the form
\begin{equation*}
S(e) = \underbrace{\langle e, Qe \rangle}_{\substack{ \text{free part} \\ \text{(kinetic +} \\ \text{mass terms)} }}
+ \underbrace{I(e)}_{\substack{ \text{interaction} \\ \text{part (cubic} \\ \text{or higher)} }}.
\end{equation*}

\begin{example}
  Why are the cubic and higher order terms called interaction terms?
For electromagnetism on a manifold $M$ we have a space of fields
  $\Fcal = \Omega^1(M) \oplus \Omega^0(M, S)$ in degree $0$. Let $F = \drm A$ and define
  \begin{equation*}
    S(A, \psi) = \int_M
    \underbrace{F \wedge \hodge F
    + \langle \psi, \slashed{\drm} \psi \rangle \dd \mathrm{vol}}_{\text{quadratic terms}}
    + \underbrace{\langle \psi, \slashed{A} \psi \rangle \dd \mathrm{vol}}_{\text{interaction terms}}.
  \end{equation*}
  Computing the Euler-Lagrange equations we obtain the system of differential equations
  \begin{align*}
    \hodge \drm \hodge F &= \bar{\psi} \gamma^\mu \psi \dd x_\mu \\
    \slashed{d}_A \psi &= 0
  \end{align*}
  which is coupled because of the interaction term.
\end{example}

% ╭──────────────────────────────────────────────────────────╮
% │             Quantization in the BV formalism             │
% ╰──────────────────────────────────────────────────────────╯
\section{Quantization in the BV formalism}

The slogan of quantization in the BV formalism is to \textit{deform the differential}.
In the perturbative context we work in formal power series in $\hbar$, for example, over the ring $\Rbb \formalpow{\hbar}$.
Quantization results in a cochain complex
$(\Oscr(\Ecal)\formalpow{\hbar}, \{S^q, \cdot\} + \hbar \laplace)$, where $\laplace$ is called the BV Laplacian, and the function
$S^q \in \Oscr(\Ecal)[[\hbar]]$ satisfies the \textbf{quantum master equation} (QME)
\begin{equation*}
  (\{S^q, \cdot \} + \hbar \laplace)^2 = 0.
\end{equation*}

\begin{example}
  In finite dimensions, consider the BV fields
  \begin{equation*}
    \Ecal^\bullet =
    (\begin{tikzcd}[sep=small]
      \Rbb \arrow[r] & \Rbb
    \end{tikzcd}).
  \end{equation*}
  concentrated in degree $0$ and $1$.
  Then
  \begin{equation*}
    \Oscr(\Ecal) \cong \Rbb \bigl[ x^1, \dots, x^n, \xi^1, \dots, \xi^n \bigr]
  \end{equation*}
  and the BV Laplacian takes the form
  \begin{equation*}
    \laplace = \sum_{\mu = 1}^n \frac{\partial}{\partial \xi^\mu} \frac{\partial}{\partial x^\mu}.
  \end{equation*}
  Note that $\laplace$ is a differential operator of degree $1$ such that $\laplace^2 = 0$.
\end{example}
The quantum action is a function of the form
\begin{equation*}
  S^q(e) = \langle e, Qe \rangle
  + I^q(e)
\end{equation*}
where $I^q \in \Oscr(\Ecal) [[\hbar]]$ is cubic mod $\hbar$ and satisfies the QME
\begin{equation*}
  Q I^q + \frac{1}{2} \{I^q, I^q\} + \hbar \laplace I^q = 0
\end{equation*}
which resembles the \textbf{Maurer-Cartan} (MC) equation. In infinite dimensions, some problems arise:

\begin{enumerate}[i)]
  \item there may be no solution to this equation. In this case we say that quantization is obstructed (there is an anomaly);
  \item the QME in infinite dimensions is ill-defined. Some functional analysis is needed to make sense of this problem.
\end{enumerate}

\chapter{Lecture 2}

In this lecture we consider a naive example illustrating how the Euler-Lagrange equations lead us to classical BV theories.

\begin{example}
  Let $\Fcal$ be a finite-dimensional vector space encoding the naive space of fields and consider an action
  \begin{equation*}
    S \colon \Fcal \longrightarrow \Rbb.
  \end{equation*}
  We say that $S$ is a naive action because adding additional terms might be necessary to guarantee that it satisfies the CME.
  The solutions to the Euler-Lagrange equation are fields $f \in \Fcal$ such that ${\drm S} (f) = 0$.

  In general, if $\Fcal = M$ for some finite-dimensional manifold $M$, we say that critical points of the action form the \textbf{critical locus} of $S$
  \begin{equation*}
    \Crit (S) = \bigl\{ p \in M \bigm| \drm S (p) = 0 \bigr\}.
  \end{equation*}
  Alternatively, we can characterize the critical locus of $S$ as an intersection in $T^* M$
  \begin{equation*}
    \Crit (S) = \Graph( \drm S) \cap \Graph (M)
  \end{equation*}
  where we identify $M$ with the zero section. It follows that
  \begin{equation*}
    \Oscr( \Crit (S) ) =
    \Oscr( \Graph (\drm S ) ) \otimes_{\Oscr(T^* M)} \Oscr(M).
  \end{equation*}
  We are going to consider a derived version of this construction, where the tensor product $\otimes$ is replaced by a derived tensor product $\otimes^{\Lbb}$. This raises the obvious questions:
  \begin{figure}[ht]
    \includestandalone{bad_intersections}
    \centering
    \caption{Well-behaved (in green) and badly-behaved (in red) intersection points.}
    \label{fig:bad_intersections}
  \end{figure}
  \begin{itemize}
    \item \textbf{Why?} The intersection might not be well-behaved, in the sense that $\drm S$ and the zero section might not intersect transversally, or even smoothly, at every point, as illustrated in figure \ref{fig:bad_intersections}. The derived approach allows us to study these badly-behaved points using Serre's intersection formula.
    \item \textbf{How?} We replace $\Oscr (\Graph(\drm S)) \otimes_{\Oscr (T^*M)} \Oscr(M)$ with a dg commutative algebra $A$ such that
    \begin{equation*}
      \Hrm^0 A = \Oscr (\Graph(\drm S)) \otimes_{\Oscr (T^*M)} \Oscr(M).
    \end{equation*}
    To compute the derived tensor product $\otimes^{\Lbb}$ we need to resolve either $\Oscr(M)$ or $\Oscr(\Graph(\drm S))$ in $\Oscr(T^* M)$-modules. Let us make use of Darboux coordinates to write
    \begin{align*}
      \Oscr(\Graph(\drm S)) &=
      \bigslant{\Oscr(T^* M)}{ \bigl( f \vert_{\Graph(\drm S)} = 0 \bigr) } \\
                            &= \bigslant{ \Oscr( T^* M )}{ \bigl( p_{\mu} - \partial_\mu S \bigr) }.
    \end{align*}
    Consider the resolution
    \begin{equation*}
      \begin{tikzcd}[sep=large]
        \dots \arrow[r] &
        \Oscr(T^* M) (\xi_1, \dots, \xi_n) \arrow[r, "\xi_{\mu} \mapsto p_{\mu} - \partial_{\mu} S"] &
        \Oscr(T^* M) \arrow[d] \\ &&
        \Oscr(\Graph(\drm S))
      \end{tikzcd}
    \end{equation*}
    which we extend to the left as a Koszul complex
    $K^{-p} = \bigwedge_{\Oscr(T^* M)}^p (\xi_1, \dots, \xi_n)$
    with differential 
    \begin{equation*}
      \drm = \sum_{\mu} (p_{\mu} - \partial_{\mu} S) \frac{\partial}{\partial \xi_{\mu}}.
    \end{equation*}
    This complex freely resolves $\Oscr(\Graph(\drm S))$. Alternatively, $(K^\bullet, \drm)$ admits a coordinate free description where
    \begin{equation*}
      K^{-p} = \Oscr(T^* M) \otimes_{\Oscr(M)} \Xfrak^p (M).
    \end{equation*}
    A model for $\Oscr(\Graph(\drm S)) \otimes_{\Oscr(T^* M)}^{\Lbb} \Oscr(M)$ is given by
    \begin{equation*}
      \Oscr(\dCrit(S)) = K^{-\bullet} \otimes_{T^* M} \Oscr(M)
    \end{equation*}
    which we call the \textbf{derived critical locus}.
    Now note that
    \begin{equation*}
      \Oscr(T^* M) \otimes_{\Oscr(M)} \PV^\bullet(M) \otimes_{\Oscr(T^* M)} \Oscr(M) \cong \PV^\bullet(M)
    \end{equation*}
    where $\PV^\bullet(M)$ denotes the complex of polyvector fields on M. The differential is given by contracting with $\drm S$ so we write
    \begin{equation*}
      \Oscr(\dCrit(S)) = (\PV^\bullet(M), -\iota_{\drm S}).
    \end{equation*}
  \end{itemize}
\end{example}

\chapter{Lecture 3}

We want to sketch how to go from the Yang-Mills action
\begin{equation*}
  S^{\text{naive}} (A) = \int_{M^n} \tr (F_A \wedge \star F_A)
\end{equation*}
to the Yang-Mills classical BV theory
\begin{equation*}
  \begin{tikzcd}
    \underbrace{\Omega^0 (M, \gfrak)}_{\text{ghosts}}^{-1}
    \arrow[r, "\drm"] &
    \underbrace{\Omega^1 (M, \gfrak)}_{\text{fields}}^{0}
    \arrow[r, "\drm \star \drm"] &
    \underbrace{\Omega^{n-1} (M, \gfrak)}_{\text{antifields}}^{1}
    \arrow[r, "\drm"] &
    \underbrace{\Omega^n (M, \gfrak)}_{\text{antighosts}}^{2}
  \end{tikzcd}
\end{equation*}
with BV action
\begin{equation*}
  S^{\text{BV}}(e) = \langle e, Q e \rangle + I(e)
\end{equation*}
where
\begin{equation*}
  \langle e, f \rangle = \int_{M^n} \tr (e \wedge f)
\end{equation*}
is the $-1$-shifted symplectic structure. There are some points to motivate:
\begin{enumerate}[i)]
  \item \textbf{fields} $\rightsquigarrow$ \textbf{fields and antifields:} coming from the derived critical locus $\text{dCrit}(S)$;
  \item \textbf{ghosts:} coming from taking the derived coinvariants of $\gfrak \curvearrowright V$.
\end{enumerate}

For Yang-Mills spacetime is a manifold $M^n$ and $\Omega^1(M, \gfrak)$ is the space of fields.
However, in what follows, let $M$ be the space of fields.
Recall that
\begin{align*}
  \Crit(S) &= \bigl\{ p \in M \bigm| \drm S_p = 0 \bigr\} \\
           &= \Graph (\drm S) \cap M
\end{align*}
in $T^* M$. Dually
\begin{equation*}
  \Oscr \bigl( \Crit(S) \bigr)
  = \Oscr \bigl( \Graph (\drm S) \bigr) \otimes_{\Oscr(T^* M)} \Oscr(M).
\end{equation*}
By homological yoga, taking the derived intersection means that we replace the tensor product $\otimes$ with the derived tensor product $\otimes^{\Lbb}$.
To find $\dCrit (S)$ we resolve either $\Oscr \bigl( \Graph (\drm S) \bigr)$ or $\Oscr(M)$ as $\Oscr(T^* M)$-modules. 
Last time, we wrote the Koszul complex
\begin{equation*}
  K^{-p} = \PV^p (M) \otimes_{\Oscr(M)} \Oscr (T^* M)
\end{equation*}
where $\PV^p = \bigwedge^p \Xfrak(M)$ and differential
\begin{equation*}
  Q : v_1 \wedge \dots \wedge v_k \otimes 1 \longmapsto
  \sum_{i=1}^k (-1)^{i+1} v_1 \wedge \dots \wedge \hat{v}_i \wedge \dots \wedge v_k
  \otimes \bigl( p(v_i) - \drm S(v_i) \bigr)
\end{equation*}

\begin{exercise}
  Check that $\Hrm^0 (K^{\bullet}, Q) \cong \Oscr \bigl( \Graph (\drm S) \bigr)$, so
  \begin{equation*}
    \dCrit (S) \cong K^{\bullet} \otimes_{\Oscr(T^* M)} \Oscr (M) \cong \PV^{\bullet} (M)
  \end{equation*}
  and thus $\Oscr \bigl( \dCrit(S) \bigr) \simeq (\PV^\bullet, - \iota_{\drm S})$.
\end{exercise}

\begin{exercise}
   Show that $\Hrm^0 \Oscr( \dCrit ) \cong \Oscr (\Crit)$.
\end{exercise}

We can enhance $\Oscr \bigl( \dCrit (S) \bigr)$ to a sheaf on $M$. Following Grothendieck
\begin{equation*}
  \dCrit(S) = \bigl( M, \PV_M^\bullet, - \iota_{\drm S} \bigr)
\end{equation*}
is an example of a \textbf{dg manifold}.

\begin{definition}
  A dg manifold is a smooth manifold $M$ with a sheaf $\Oscr_M$ of \textbf{dg commutative algebras} (DGCAs) locally isomorphic to $\Oscr_M(U) \cong \bigwedge \Ecal (U)$, where $\Ecal$ are the smooth sections of $E \rightarrow M$.
\end{definition}

Ignoring the differential, we get a sheaf $\bigl(M, \PV_M^\bullet\bigr)$ on $M$ such that
\begin{equation*}
  \PV_M = \bigwedge \Xfrak_M \cong \Sym \Xfrak[1].
\end{equation*}
The underlying graded manifold of $\dCrit(S)$ is
\begin{equation*}
  T^*[-1]M = \bigl( M, \Sym \Xfrak[1] \bigr)
\end{equation*}
displaying the following properties:
\begin{enumerate}[i)]
  \item the graded manifold $T^*[-1]M$ is a $-1$-shifted symplectic graded manifold just as $T^*M$ is a $0$-shifted symplectic manifold;
  \item Induced from the $-1$-shifted symplectic structure we get a $1$-shifted Poisson bracket on $\Oscr \bigl( T^* [-1] M \bigr) = \PV(M)$ known as the \textbf{Schouten bracket}
  \begin{align*}
    \{ f, g \} &= 0, \\
    \{v, f \} &= v f, \\
    \{ v, w \} &= [v, w], \\
    \{u, v \wedge w \} &= \{u, v\} \wedge w + v \wedge \{ u, w \}
  \end{align*}
for $f, g \in \Oscr(M)$ and $u, v, w \in \PV^{-1}(M)$.
\end{enumerate}

\begin{exercise}
  Show that $-\iota_{\drm S} = \{ S, \cdot \}$.
\end{exercise}

\begin{definition}
  A $\Pbb_0$ algebra $\bigl( A, \drm, \{ \cdot, \cdot \} \bigr)$ is a DGCA $(A, \drm)$ equipped with a $-1$-shifted Poisson bracket $\{ \cdot, \cdot \}: A \otimes A \rightarrow A$ obeying:
  \begin{enumerate}[i)]
    \item \textbf{graded skew-symmetry:}
      \begin{equation*}
        \{x, y\} = - (-1)^{(|x|+1)(|y|+1)} \{y, x\};
      \end{equation*}
    \item \textbf{graded Poisson identity}:
      \begin{equation*}
        \{x, yz \} = \{x, y\} z + (-1)^{(|x|+1)|y|} y \{x, z\}
      \end{equation*}
      so $\{x, \cdot \}$ is a degree $|x|+1$ derivation;
    \item \textbf{graded Jacobi identity}:
      \begin{equation*}
        \{x, \{y, z\} \} = \{ \{x, y\}, z \} + (-1)^{(|x|+1)(|y|+1)} \{y, \{x, z\} \};
      \end{equation*}
    \item \textbf{compatibility with differential:}
      \begin{equation*}
        \drm \{x, y\} = \{ \drm x, y \} + (-1)^{|x|+1} \{x, \drm y \}.
      \end{equation*}
  \end{enumerate}
\end{definition}

\begin{exercise}
  Check that the Schouten bracket defines a $\Pbb_0$-algebra on $\Oscr\bigl(T^*[-1] M \bigr)$.
\end{exercise}

\chapter{Lecture 5}

(John Huerta)

\section{Ultimate Goal}

Define and use the Feynman ``path'' integral
\begin{equation*}
  \int_{\phi \in \Fcal} e^{-\frac{S (\phi)}{\hbar}} \, D\phi .
\end{equation*}
(Euclidean field theory)

In the constructive track: see Gonçalo on how to do this. In the BV
track: we will produce a formal power series in $\hbar$.

\section{Recall}

\begin{itemize}
\item From now on: We work {\em perturbatively}, i.e., {\em formally}
  (in Algebraic Geometry speak), i.e., in {\em formal power series},
  i.e., {\em infinitesimally}.
\item Now $M$ is going to be a finite dimensional manifold, denoting space-time. E.g.,
  \[
    M = \Rbb^d, \mbox{ or } M = \mathrm{pt}.
  \]
\item $\Fcal$ always denotes the naive fields, a sheaf of vector
  spaces on $M$; specifically, sections of some vector bundle
  $F\longrightarrow M$.

  Example: Yang-Mills fields for a trivial $G$-bundle
  $M\times G \longrightarrow M$, then
  $\Fcal (M)=\Omega^1 (M,\gfrak)$, where
  $\gfrak=\mathrm{Lie} (G)$.
\item $\Ecal$ (``extended''), the space of BV-fields, a sheaf of {\em
    graded} vector spaces over $M$, sections of a graded vector bundle
  $E\longrightarrow M$. $\Ecal^0 (M)=\Fcal (M)$.

  In the Yang-Mills example, where $d=\dim M$
  \begin{equation*}
    \Ecal(M) = \underbrace{\Omega^0 (M,\gfrak)}_{\text{``ghosts''}}^{-1}
    \oplus\underbrace{\Omega^1(M,\gfrak)}_{\text{``fields''}}^{0}
    \oplus\underbrace{\Omega^{d-1} (M,\gfrak)}_{\text{``anti-fields''}}^{1}
    \oplus\underbrace{\Omega^d(M,\gfrak)}_{\text{``anti-ghosts''}}^{2}
  \end{equation*}
\end{itemize}

\section{BV Formulation of Gauge Theory}

Input: Naive gauge theory
\begin{equation*}
\underbrace{\Lcal}_{
  \substack{\text{Lie algebra of}\\
    \text{infinitesimal gauge}\\
    \text{transformations}} } \curvearrowright
\underbrace{\Fcal}_{\text{space of naive fields}}\,.
\end{equation*}
The action may be non-linear. In the Young-Mills example it is an
affine action
\begin{equation*}
  \Omega^0 (M\gfrak)\curvearrowright\Omega^1 (M\gfrak)\,.
\end{equation*}
There is a two-step process to writing down the gauge theory:
\begin{enumerate}
\item Take the {\em ``stacky quotient''}
  \begin{equation*}
    \Fcal \rightsquigarrow \Fcal \hquot \Lcal \qquad \text{(this lecture)}
  \end{equation*}
\item Take the derived critical locus of $S_\text{gauge}$:
  \begin{equation*}
    T^*[-1]\left(\Fcal\hquot\Lcal\right)\,. \qquad \text{(already done)}
  \end{equation*}
\end{enumerate}

\section{Lightning Fast Introduction to Derived Invariants}
$\gfrak$ a finite dimensional Lie algebra, $R$ a finite dimensional
representation of $\gfrak$
\begin{equation*}
  \gfrak\rightarrow\mathfrak{gl} (R)
\end{equation*}
over some field $\kbold\in\{\Rbb,\Cbb\}$.
Observe that
\begin{align*}
  R^{\gfrak}&=\{v\in \Rbb \mid Xv=0 \text{ for all } X\in\gfrak \}\\
          &=\mathrm{Hom}_{\gfrak} (\kbold,R)
\end{align*}
Derived version $\mathrm{Hom} \rightsquigarrow \Rbb\mathrm{Hom}$.

Try $R^{\mathrm{h}\gfrak}=\Rbb\mathrm{Hom}_{U\gfrak} (\kbold,R)$, where
$U$ is the enveloping algebra. I.e.,
\begin{equation*}
  U\gfrak=\frac{T\gfrak}{x\otimes{}y-y\otimes{}x-[x,y]}
\end{equation*}
where $T\gfrak$ is the tensor algebra.
\begin{fact}
  $\mathrm{Rep_{\gfrak}}\simeq U\gfrak\mathrm{-mod}\,.$
\end{fact}
To compute $\Rbb\mathrm{Hom}_{U\gfrak} (\kbold,R)$ we need to resolve $\kbold$
or $R$ as $U\gfrak$ modules.

Similar to the Koszul complex
\begin{equation*}
    \begin{tikzcd}
    \cdots \arrow[r] &
    \overset{-k}{\Lambda^k\gfrak\otimes U\gfrak} \arrow[r] &
    \cdots \arrow[r] &
    \overset{-1}{\gfrak\otimes U\gfrak} \arrow[r] &
    \overset{0}{U\gfrak}
  \end{tikzcd}
\end{equation*}
with differential
\begin{align*}
  \Lambda^{k+1}\gfrak\otimes\longrightarrow&\Lambda^k\otimes{}U\gfrak\\
  x_0\wedge\cdots\wedge{}x_k\otimes y \longmapsto& \sum_{i=0}^k (-1)^i x_0\wedge\cdots\widehat{x_i}\cdots\wedge x_k \otimes x_i y\\
  +&\sum_{i<j} (-1)^{i+j} [x_i,x_j]\wedge x_0\wedge\cdots\widehat{x_i}\cdots\widehat{x_j}\cdots\wedge x_k \otimes y\,.
\end{align*}

With this differential
\begin{align*}
  H^0 \left(\Lambda^\bullet\gfrak\otimes U\gfrak\right)&\simeq \kbold\\ %%% use sideset
  H^k \left(\Lambda^\bullet\gfrak\otimes U\gfrak\right)&=0 \qquad \text{for $k<0$}
\end{align*}
Hence
\begin{align*}
  R^{\mathrm{h}\gfrak}&=\Rbb\mathrm{Hom}_{U\gfrak} (\kbold, R)\\
                      &=\mathrm{Hom}_{U\gfrak} (\Lambda^{\bullet}\gfrak\otimes U\gfrak, R)\\
                      &\simeq\mathrm{Hom}_{\kbold} (\Lambda^{\bullet}\gfrak, R)
\end{align*}
because $\Lambda^\bullet\gfrak\otimes{}U\gfrak$ is free.

\begin{definition}
  For $\gfrak$ a Lie algebra, $R$ a representation of $\gfrak$, the
  {\em Chevalley-Eilenberg complex} $C^\bullet (\gfrak,R)$ is defined as
  \begin{equation*}
    C^k (\gfrak,R)=\mathrm{Hom} (\Lambda^k\gfrak,R)
  \end{equation*}
  with
  \begin{align*}
    \dd\omega (x_0,\ldots,x_k)&=\sum_{i=0}^{k} (-1)^i x_i\cdot\omega (x_0,\ldots,\widehat{x_i},\ldots,x_k)\\
                              &+\sum_{i<j} (-1)^{i+j} \omega ([x_i,x_j],x_0,\ldots,\widehat{x_i},\ldots,\widehat{x_j},\ldots,x_k)
  \end{align*}
\end{definition}

Conclusion. Back to $R=\Ocal (V)$, then
\begin{align*}
  R^{\mathrm{h}\gfrak}&=\Ocal (V)^{\mathrm{h}\gfrak}\\
                      &=\mathrm{Hom}_{\kbold} (\Lambda^\bullet\gfrak,\Ocal (V))\\
                      &\simeq\Lambda^\bullet\gfrak^*\otimes\Ocal (V)\\
                      &\simeq\Sym (\gfrak^*[-1])\otimes\Sym (V^*)\\
                      &\simeq\Sym (\overset{0}{V^*}\oplus\overset{1}{\gfrak^*[-1]})\\
                      &\simeq\Ocal (\overset{-1}{\gfrak[1]}\oplus \overset{0}{V})\\
                      &=:\Ocal (V\hquot\gfrak)\,.
\end{align*}
Hence
\begin{definition}
  $V\hquot\gfrak:=\gfrak[1]\oplus{}V$
\end{definition}

Puzzle: what happened to the differential
$\dd$. It becomes a vector field on $\gfrak\oplus{}V$!.

\section{Back to Yang-Mills}

\begin{align*}
  V&\rightsquigarrow\Omega^1 (M,\gfrak)\\
  \gfrak&\rightsquigarrow\Omega^0 (M,\gfrak)
\end{align*}

Step 1: $\Omega^1 (M,\gfrak)\hquot\Omega^0 (M,\gfrak):=\Omega^0 (M,\gfrak)[1]\oplus\Omega^1 (M,\gfrak)$

Step 2: $\Ecal$ for Yang-Mills
\begin{align*}
  T^*[-1] (\overset{-1}{\Omega^0 (M,\gfrak)[1]}\oplus{}\overset{0}{\Omega^1 (M,\gfrak)})
  &\simeq\Omega^0 (M,\gfrak)[1]\oplus\Omega^1 (M,\gfrak)\\
  &\;\oplus (\Omega^0 (M,\gfrak)[1]\oplus\Omega^1 (M,\gfrak))^*[-1]\\
  &\simeq\Omega^0 (M,\gfrak)[1]\oplus\Omega^1 (M,\gfrak)\\
  &\;\oplus\Omega^{d-1} (M,\gfrak)[-1]\oplus\Omega^d (M,\gfrak)[-2]
\end{align*}

\sloppy
\printbibliography
\end{document}
