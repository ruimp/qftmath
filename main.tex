%! TeX program = lualatex
% ╭──────────────────────────────────────────────────────────╮
% │      This document should be compiled with LuaLaTeX      │
% ╰──────────────────────────────────────────────────────────╯
\documentclass[a4paper, 12pt, oneside]{memoir}

% ── Style ─────────────────────────────────────────────────────────────
\setulmarginsandblock{2.25cm}{2cm}{*}
\setlrmarginsandblock{3.25cm}{3.25cm}{*}
\checkandfixthelayout

\chapterstyle{dash}
\renewcommand*{\chaptitlefont}{\normalfont\Huge\bfseries\sffamily}
\renewcommand*{\chapnamefont}{\normalfont\huge\bfseries\sffamily}
\renewcommand*{\chapnumfont}{\normalfont\huge\bfseries\sffamily}
\setsecheadstyle{\normalfont\Large\bfseries\sffamily}
%\renewcommand{\chaptername}{Lecture}

% ── Language ──────────────────────────────────────────────────────────
\usepackage{polyglossia}
\setdefaultlanguage[variant=usmax]{english}

% ── Fonts ─────────────────────────────────────────────────────────────
\usepackage[libertinus, upint, vvarbb]{newtx}

% ── Tikz ──────────────────────────────────────────────────────────────
\usepackage{standalone}
\usepackage{tikz}
\usetikzlibrary{cd}
\tikzset{>={Straight Barb[scale=.8]}}
\tikzcdset{arrow style=tikz, diagrams={>={Straight Barb[scale=.8]}}}

% ── Bibliography ──────────────────────────────────────────────────────
\usepackage{hyperref}
\usepackage[backend=biber, style=alphabetic]{biblatex}
\addbibresource{library.bib}

% ── Tables ────────────────────────────────────────────────────────────
\usepackage{booktabs}
\usepackage{makecell}

% ── Theorem environments ──────────────────────────────────────────────
\usepackage{mathtools}
\usepackage[hyperref, thmmarks]{ntheorem}

\theoremheaderfont{\sffamily\bfseries}
\theorembodyfont{\upshape}
\theoremseparator{.}
\theoremsymbol{}
\newtheorem{theorem}{Theorem}
\newtheorem{definition}{Definition}
\newtheorem{proposition}{Proposition}
\newtheorem{example}{Example}
\newtheorem{exercise}{Exercise}
\newtheorem{fact}{Fact}
\theoremstyle{nonumberplain}
\theoremheaderfont{\sffamily\bfseries}
\newtheorem{remark}{Remark}
\theoremsymbol{\ensuremath{\blacksquare}}
\newtheorem{proof}{Proof}

% ── Symbols ───────────────────────────────────────────────────────────
\usepackage{slashed}

% ── Quotients ─────────────────────────────────────────────────────────
\usepackage{faktor}

% ── Macros ────────────────────────────────────────────────────────────
\newcommand{\Escr}{\mathscr{E}}
\newcommand{\Oscr}{\mathscr{O}}

\newcommand{\Ecal}{\mathcal{E}}
\newcommand{\Fcal}{\mathcal{F}}
\newcommand{\Gcal}{\mathcal{G}}
\newcommand{\Lcal}{\mathcal{L}}
\newcommand{\Ocal}{\mathcal{O}}
\newcommand{\Tcal}{\mathcal{T}}

\newcommand{\Xfrak}{\mathfrak{X}}
\newcommand{\gfrak}{\mathfrak{g}}

\newcommand{\Rbb}{\mathbb{R}}
\newcommand{\Cbb}{\mathbb{C}}
\newcommand{\Lbb}{\mathbb{L}}
\newcommand{\Pbb}{\mathbb{P}}

\newcommand{\kbold}{\pmb{k}}

\newcommand{\drm}{\mathrm{d}}
\newcommand{\hrm}{\mathrm{h}}
\newcommand{\Crm}{\mathrm{C}}
\newcommand{\Drm}{\mathrm{D}}
\newcommand{\Hrm}{\mathrm{H}}
\newcommand{\dd}{\, \mathrm{d}}
\newcommand{\DD}{\, \mathrm{D}}

\DeclareMathOperator{\tr}{tr}
\DeclareMathOperator{\Aut}{Aut}
\DeclareMathOperator{\Map}{Map}
\DeclareMathOperator{\Crit}{Crit}
\DeclareMathOperator{\Graph}{Graph}
\DeclareMathOperator{\PV}{PV}
\DeclareMathOperator{\Sym}{Sym}
\DeclareMathOperator{\Lie}{Lie}
\DeclareMathOperator{\Der}{Der}
\DeclareMathOperator{\Dens}{Dens}
\DeclareMathOperator{\hquot}{/\mkern-6mu/}

% ── hodge star ────────────────────────────────────────────────────────
\newcommand{\hodge}{{\star}}
% ── big quotient ──────────────────────────────────────────────────────
\newcommand{\bigslant}[2]{{\left.\raisebox{.2em}{$#1$}\middle/\raisebox{-.2em}{$#2$}\right.}}
% ── homotopy quotient ─────────────────────────────────────────────────
\newcommand*{\doublequotient}[2]{
  \raisebox{0.5\height}{\ensuremath{#1}}
  \mkern-5mu\diagup\mkern-12mu\diagup\mkern-4mu
  \raisebox{-0.5\height}{\ensuremath{#2}}
}
% ── formal polynomials ────────────────────────────────────────────────
\newcommand{\formalpow}[1]{[\![ {#1} ]\!]}
% ── homotopy critical locus ───────────────────────────────────────────
\newcommand{\dCrit}{\Crit^{\mathrm{h}}}
% ── group action ──────────────────────────────────────────────────────
\newcommand{\acts}{ \ \rotatebox[origin=c]{-90}{$\circlearrowright$} \ }

% ── Title Page ────────────────────────────────────────────────────────
\title{\sffamily\bfseries Quantum Field Theory for Mathematicians}
\author{}
\date{}

\begin{document}
\maketitle
\firmlists

% ── Setup tikz colors ─────────────────────────────────────────────────
% ╭──────────────────────────────────────────────────────────╮
% │        Color settings for LaTeX and TikZ figures         │
% ╰──────────────────────────────────────────────────────────╯

% ╭──────────────────────────────────────────────────────────╮
% │                    Foreground Colors                     │
% ╰──────────────────────────────────────────────────────────╯
% ── Light yellow ──────────────────────────────────────────────────────
\definecolor{yellow1}{HTML}{FEEBC6}

% ── Blue ──────────────────────────────────────────────────────────────
\definecolor{blue1}{HTML}{C3EAF0}

% ── Green ─────────────────────────────────────────────────────────────
\definecolor{green1}{HTML}{C3E2D8}

% ── Red ───────────────────────────────────────────────────────────────
\definecolor{red1}{HTML}{D98485}

% ── Pink ──────────────────────────────────────────────────────────────
\definecolor{pink1}{HTML}{CC9DC7}

% ╭──────────────────────────────────────────────────────────╮
% │                    Background colors                     │
% ╰──────────────────────────────────────────────────────────╯
% ── Dark yellow ───────────────────────────────────────────────────────
\definecolor{yellow2}{HTML}{E0CFAE}


% ── Content ───────────────────────────────────────────────────────────
\chapter{Lecture 1}

\section{History}

\begin{center}
  \begin{tabular}{ccccc}
    Date & People & What & Why & Techniques \\
    \midrule
    1969 &
    \makecell{ Faddeev \\ and Popov } &
    \makecell{ Gauge fixing \\ (adding ghosts) } &
    \makecell{ Quantize \\ Yang-Mills } &
    \makecell{ Berezinian \\ integration } \\
    \addlinespace
    1973 & 
    \makecell{ 't Hooft and \\ Veltman } &
    \makecell{ Quantized \\ Yang-Mills } &
    \makecell{ Quantize \\ Yang-Mills } &
    \makecell{ Feynman \\ diagrams } \\
    \addlinespace
    1975 &
    \makecell{ Becchi, Rouet, \\ Stora, Tyutin \\ (BRST) } &
    \makecell{ Cohomological \\ theory to quantize \\ Yang-Mills } &
    \makecell{ Understanding \\ 't Hooft \\ and Veltman } &
    \makecell{ Derived invariants \\ (Lie algebra \\ cohomology) } \\
    \addlinespace
    1981 &
    \makecell{ Batallin and \\ Vilkovisky \\ (BV) } &
    \makecell{ Quantize systems \\ with complicated \\ gauge symmetries } &
    Supergravity &
    \makecell{ Derived \\ intersections \\ (Koszul complexes) } \\
    \addlinespace
    1992 & Henneaux &
    \makecell{ Quantize \\ Yang-Mills \\ using BV } &
    \makecell{ Analyze \\ Yang-Mills \\ using BV } &
    \makecell{ Derived \\ intersections} \\
    \addlinespace
    2007 & Costello &
    \makecell{ Combine BV \\ with effective \\ field theory } &
    \makecell{ Make BV \\ quantization \\ rigorous } &
    \makecell{ Derived everything, \\ analysis, and \\ homotopy theory }
  \end{tabular}
\end{center}

\section{References}

The main references for this seminar will be:

\begin{itemize}
  \item Costello - Renormalization and Effective Field Theory \cite{costelloRenormalizationEffective};
  \item Elliot, Williams, Yoo - Asymptotic Freedom in the BV Formalism \cite{elliottAsymptoticFreedom};
  \item Gwilliam - Factorization algebras and free field theories \cite{gwilliamFactorizationAlgebras}.
\end{itemize}

\section{Roadmap to BV Quantization}

\begin{figure}
  \begin{tikzpicture}
  \draw[thick, domain=-7:7, smooth, variable=\x] plot ( {\x}, {0.15*sin(deg(\x+2.2))} );
  \node at (-6, 1.2) {
    \begin{tabular}{c}
      Classical \\
      BV theory
    \end{tabular}
  };
  \draw[thick, ->] (-6, .7) -- (-6, .2);
  \node at (-3, 1) {
    \begin{tabular}{c}
      Quantum \\
      BV theory
    \end{tabular}
  };
  \draw[thick, ->] (-3, .5) -- (-3, 0);
  \node at (0, 1) { Quantization };
  \draw[thick, ->] (0, .7) -- (0, .2);
  \node at (2.9, 1.6) {
    \begin{tabular}{c}
      Functional \\
      analysis
    \end{tabular}
  };
  \node at (1.2, .4) { \miniscule\texttt{DANGER!} };
  \draw[thick] (.7, .25) rectangle ++(.95, .3);
  \draw[thick] (1.2, .25) -- (1.2, -.03);
  \node at (6, 1.2) {
    \begin{tabular}{c}
      Quantum \\ 
      Yang-Mills
    \end{tabular}
  };
  \draw[thick, ->] (6, .7) -- (6, .2);
  % Mountains
  \draw[thick] (1.48, -.08) -- (1.75, 0.3);
  \draw[thick] (1.75, 0.3) -- (2.03, -.13);
  \draw[fill] (1.75, 0.3) circle (.2pt);
  \draw[semithick] (1.67, .2) -- (1.83, 0.19);
  \draw[thick] (1.87, .12) -- (2.15, 0.67);
  \draw[thick] (2.15, 0.67) -- (2.52, -.14);
  \draw[fill] (2.15, 0.67) circle (.19pt);
  \draw[semithick] (2.03, 0.43) -- (2.12, 0.5);
  \draw[fill] (2.12, 0.5) circle (.101pt);
  \draw[semithick] (2.12, 0.5) -- (2.19, 0.44);
  \draw[fill] (2.19, 0.44) circle (.101pt);
  \draw[semithick] (2.19, 0.44) -- (2.24, 0.47);
  \draw[thick] (2.28, .4) -- (2.45, 0.75);
  \draw[thick] (2.45, 0.75) -- (2.9, -.15);
  \draw[fill] (2.45, 0.75) circle (.19pt);
  \draw[semithick] (2.36, 0.58) -- (2.42, 0.55);
  \draw[fill] (2.42, 0.55) circle (.101pt);
  \draw[semithick] (2.42, 0.55) -- (2.48, 0.58);
  \draw[fill] (2.48, 0.58) circle (.101pt);
  \draw[semithick] (2.48, 0.58) -- (2.55, 0.55);
  \draw[thick] (2.53, .58) -- (2.65, 0.8);
  \draw[thick] (2.65, 0.8) -- (2.79, .57);
  \draw[fill] (2.65, 0.8) circle (.19pt);
  \draw[thick] (2.66, .32) -- (3, 1);
  \draw[thick] (3.65, -.06) -- (3, 1);
  \draw[fill] (3, 1) circle (.19pt);
  \draw[semithick] (2.89, .8) -- (2.95, .75);
  \draw[fill] (2.95, 0.75) circle (.101pt);
  \draw[semithick] (2.95, .75) -- (3, .8);
  \draw[fill] (3, 0.8) circle (.101pt);
  \draw[semithick] (3, .8) -- (3.05, 0.75);
  \draw[fill] (3.05, 0.75) circle (.101pt);
  \draw[semithick] (3.05, 0.75) -- (3.12, 0.82);
  \draw[thick] (3.3, 0.5) -- (3.45, 0.85);
  \draw[thick] (3.45, 0.85) -- (3.9, -.03);
  \draw[fill] (3.45, 0.85) circle (0.19pt);
  \draw[semithick] (3.36, .65) -- (3.43, .69);
  \draw[fill] (3.43, .69) circle (.101pt);
  \draw[semithick] (3.43, .69) -- (3.49, .65);
  \draw[fill] (3.49, .65) circle (.101pt);
  \draw[semithick] (3.49, .65) -- (3.54, .69);
  \draw[thick] (3.67, .4) -- (3.85, .65);
  \draw[thick] (3.85, 0.65) -- (4.2, 0.02);
  \draw[fill] (3.85, 0.65) circle (.19pt);
\end{tikzpicture}
\centering
\caption{Roadmap to BV quantization.}
\label{fig:roadmap}
\end{figure}

The space of fields $\Ecal^\bullet$ is a cochain complex
\begin{equation*}
  \begin{tikzcd}
    \dots \arrow[r] &
    \Ecal^{-1} \arrow[r, "Q"] &
    \Ecal^0 \arrow[r, "Q"] &
    \Ecal^1 \arrow[r, "Q"] &
    \Ecal^2 \arrow[r] &
    \dots
  \end{tikzcd}
\end{equation*}
equipped with a differential $Q$ such that $Q^2 = 0$. Moreover, $\Ecal$ admits a $-1$-shifted symplectic structure, that is, there exists a non degenerate pairing of degree $-1$
\begin{equation*}
  \langle \cdot , \cdot \rangle :
  \Ecal \otimes \Ecal \longrightarrow \Rbb [-1]
\end{equation*}
such that $\langle x, y \rangle = -(-1)^{(|x|+1)(|y|+1)} \langle y, x \rangle$. This structure defines a $+1$-shifted Poisson bracket
\begin{equation*}
  \{ \cdot, \cdot \} :
  \Ocal( \Ecal ) \otimes \Ocal( \Ecal ) \longrightarrow \Ocal( \Ecal )
\end{equation*}
where $\Ocal ( \Ecal ) \cong \mathrm{Sym}^\bullet (\Ecal^\vee)$ is the (graded) commutative algebra of polynomial functions on the dual complex $\Ecal^\vee$. Pick $S \in \Ocal (\Ecal)$ obeying the \textbf{classical master equation} (CME)
\begin{equation}
  \label{eq:cme}
  \{ S, S \} = 0.
\end{equation}
The data $(\Ecal, \langle \cdot, \cdot \rangle, S)$ defines a \textbf{classical BV theory}. The CME says $\{ S, \cdot \}$ is a differential which makes $(\Ocal ( \Ecal ), \{S, \cdot \} )$ into a cochain complex such that
\begin{equation*}
  \Hrm^0 \Ocal ( \Ecal ) \cong
  \Ocal( \mathrm{Crit} (S) ),
\end{equation*}
where $\mathrm{Crit} (S)$ denotes the critical locus of $S$. We will restrict to $S$ of the form
\begin{equation*}
S(e) = \underbrace{\langle e, Qe \rangle}_{\substack{ \text{free part} \\ \text{(kinetic +} \\ \text{mass terms)} }}
+ \underbrace{I(e)}_{\substack{ \text{interaction} \\ \text{part (cubic} \\ \text{or higher)} }}.
\end{equation*}

\begin{example}
  Why are the cubic and higher order terms called interaction terms? For electromagnetism on a manifold $M$ we have a space of fields
  $\Fcal = \Omega^1(M) \oplus \Omega^0(M, S)$ in degree $0$. Let $F = \drm A$ and define
  \begin{equation*}
    S(A, \psi) = \int_M
    \underbrace{F \wedge \hodge F
    + \langle \psi, \slashed{\drm} \psi \rangle \dd \mathrm{vol}}_{\text{quadratic terms}}
    + \underbrace{\langle \psi, \slashed{A} \psi \rangle \dd \mathrm{vol}}_{\text{interaction term}}.
  \end{equation*}
  Computing the Euler-Lagrange equations we obtain the system of differential equations
  \begin{equation*}
    \begin{cases}
      \hodge \drm \hodge F = \bar{\psi} \gamma^\mu \psi \dd x_\mu \\
      \slashed{d}_A \psi = 0
    \end{cases}
  \end{equation*}
  which is coupled because of the interaction term.
\end{example}

\section{Quantization in the BV formalism}

The slogan of quantization in the BV formalism is to \emph{deform the differential}. In the perturbative context we work in formal power series in $\hbar$, for example, over the ring $\Rbb[[\hbar]]$. Quantization results in a cochain complex
$(\Ocal(\Ecal)[[\hbar]], \{S^q, \cdot\} + \hbar \Delta)$, where $\Delta$ is called the BV Laplacian, and 
$S^q \in \Ocal(\Ecal)[[\hbar]]$ satisfies the \textbf{quantum master equation} (QME)
\begin{equation}
  \label{eq:qme}
  (\{S^q, \cdot \} + \hbar \Delta)^2 = 0
\end{equation}

\begin{example}
  In finite dimensions ($\Fcal \cong \Rbb^n$) the BV fields are
  $\Ecal = \Rbb^n \longrightarrow \Rbb^n$
  therefore
  \begin{equation*}
    \Ocal(\Ecal) \cong \Rbb [x^1, \dots, x^n, \xi^1, \dots, \xi^n]
  \end{equation*}
  and the BV Laplacian takes the form
  \begin{equation*}
    \Delta = \sum_{\mu = 1}^n \frac{\partial}{\partial \xi^\mu} \frac{\partial}{\partial x^\mu}.
  \end{equation*}
  In this form, it becomes clear that $\Delta$ is a differential operator of degree $1$ such that $\Delta^2 = 0$.
\end{example}
\begin{equation*}
  S^q(e) = \langle e, Qe \rangle
  + I^q(e)
\end{equation*}
where $I^q \in \Ocal(\Ecal) [[\hbar]]$ is cubic mod $\hbar$ and satisfies the QME
\begin{equation*}
  Q I^q + \frac{1}{2} \{I^q, I^q\} + \hbar \Delta I^q = 0
\end{equation*}
which resembles, in this form, the \textbf{Maurer-Cartan} (MC) \textbf{equation}. In infinite dimensions, some problems arise:

\begin{enumerate}
  \item there may be no solution to this equation. In this case we say that quantization is obstructed (there is an anomaly);
  \item the QME in infinite dimensions is ill-defined. Some functional analysis is needed to make sense of this problem.
\end{enumerate}

\chapter{Lecture 2}

In this lecture we consider a naive example that aims to exemplify how the Euler-Lagrange equations lead us to classical BV theories.

\begin{example}
  Let $\Fcal$ be a finite-dimensional vector space encoding the naive space of fields and consider an action
  \begin{equation*}
    S: \Fcal \longrightarrow \Rbb.
  \end{equation*}
  We say that $S$ is a naive action because it might be necessary to add additional terms to $S$ to guarantee that it satisfies the CME. The solutions to the Euler-Lagrange equation are fields $f \in \Fcal$ such that ${\drm S}_f = 0$. Restricting to the case $\Fcal = M$ for some finite-dimensional manifold $M$, we say that critical points of the action form the \textbf{critical locus} of $S$
  \begin{equation*}
    \Crit (S) = \bigl\{ p \in M \bigm| \drm S_p = 0 \bigr\}.
  \end{equation*}
  Alternatively, we can characterize the critical locus of $S$ as an intersection in $T^* M$
  \begin{equation*}
    \Crit (S) = \Graph( \drm S) \cap \Graph (M)
  \end{equation*}
  where we identify $M$ with the zero section. It follows that
  \begin{equation*}
    \Ocal( \Crit (S) ) =
    \Ocal( \Graph (\drm S ) ) \otimes_{\Ocal(T^* M)} \Ocal(M).
  \end{equation*}
  We are going to consider a derived version of this construction, where the tensor product $\otimes$ is replaced by a derived tensor product $\otimes^{\Lbb}$. This raises the obvious questions:
  \begin{figure}[ht]
    \begin{tikzpicture}
      \draw[thick, ->] (-4.5, 0) -- (4.5, 0);
      \draw[thick, domain=-3.37:3.37, smooth, variable=\x] plot ( {\x}, {.35*sin(2*deg(\x))*\x*\x-.35} );
      \draw[ultra thick, color=blue] (-3.1, 0) circle (5pt);
      \draw[ultra thick, color=blue] (-1.741, 0) circle (5pt);
      \draw[ultra thick, color=red] (1.12, 0) circle (5pt);
      \draw[ultra thick, color=blue] (3.18, 0) circle (5pt);
    \end{tikzpicture}
    \centering
    \caption{Well-behaved (in red) and badly-behaved (in red) points of an intersection.}
    \label{fig:bad_intersections}
  \end{figure}
  \begin{itemize}
    \item \textbf{Why?} This intersection might not be well behaved, in the sense that $\drm S$ and the zero section might not intersect transversally at every point, as illustrated in figure \ref{fig:bad_intersections}. The derived approach allows us to study these badly-behaved points using Serre's intersection formula.
    \item \textbf{How?} We replace $\Ocal (\Graph(\drm S)) \otimes_{\Ocal (T^*M)} \Ocal(M)$ with a dg commutative algebra $A$ such that
    \begin{equation*}
      \Hrm^0 A = \Ocal (\Graph(\drm S)) \otimes_{\Ocal (T^*M)} \Ocal(M).
    \end{equation*}
    To compute the derived tensor product $\otimes^{\Lbb}$ we need to resolve either $\Ocal(M)$ or $\Ocal(\Graph(\drm S))$ in $\Ocal(T^* M)$-modules. Let us make use of Darboux coordinates to resolve
    \begin{align*}
      \Ocal(\Graph(\drm S)) &=
      \bigslant{\Ocal(T^* M)}{ \bigl( f \vert_{\Graph(\drm S)} = 0 \bigr) } \\
                            &= \bigslant{ \Ocal( T^* M )}{ \bigl( p_{\mu} - \partial_\mu S \bigr) }.
    \end{align*}
  \end{itemize}
\end{example}


\chapter{Lecture 3}

We want to sketch how to go from the Yang-Mills action
\begin{equation*}
  S^{\text{naive}} (A) = \int_{M^n} \tr (F_A \wedge \star F_A)
\end{equation*}
to the Yang-Mills classical BV theory
\begin{equation*}
  \begin{tikzcd}
    \underbrace{\Omega^0 (M, \gfrak)}_{\text{ghosts}}^{-1}
    \arrow[r, "\drm"] &
    \underbrace{\Omega^1 (M, \gfrak)}_{\text{fields}}^{0}
    \arrow[r, "\drm \star \drm"] &
    \underbrace{\Omega^{n-1} (M, \gfrak)}_{\text{antifields}}^{1}
    \arrow[r, "\drm"] &
    \underbrace{\Omega^n (M, \gfrak)}_{\text{antighosts}}^{2}
  \end{tikzcd}
\end{equation*}
with BV action
\begin{equation*}
  S^{\text{BV}}(e) = \langle e, Q e \rangle + I(e)
\end{equation*}
where
\begin{equation*}
  \langle e, f \rangle = \int_{M^n} \tr (e \wedge f)
\end{equation*}
is the $-1$-shifted symplectic structure. There are some points to motivate:
\begin{enumerate}[i)]
  \item \textbf{fields} $\rightsquigarrow$ \textbf{fields and antifields:} coming from the derived critical locus $\text{dCrit}(S)$;
  \item \textbf{ghosts:} coming from taking the derived coinvariants of $\gfrak \curvearrowright V$.
\end{enumerate}

For Yang-Mills spacetime is a manifold $M^n$ and $\Omega^1(M, \gfrak)$ is the space of fields.
However, in what follows, let $M$ be the space of fields.
Recall that
\begin{align*}
  \Crit(S) &= \bigl\{ p \in M \bigm| \drm S_p = 0 \bigr\} \\
           &= \Graph (\drm S) \cap M
\end{align*}
in $T^* M$. Dually
\begin{equation*}
  \Oscr \bigl( \Crit(S) \bigr)
  = \Oscr \bigl( \Graph (\drm S) \bigr) \otimes_{\Oscr(T^* M)} \Oscr(M).
\end{equation*}
By homological yoga, taking the derived intersection means that we replace the tensor product $\otimes$ with the derived tensor product $\otimes^{\Lbb}$.
To find $\dCrit (S)$ we resolve either $\Oscr \bigl( \Graph (\drm S) \bigr)$ or $\Oscr(M)$ as $\Oscr(T^* M)$-modules. 
Last time, we wrote the Koszul complex
\begin{equation*}
  K^{-p} = \PV^p (M) \otimes_{\Oscr(M)} \Oscr (T^* M)
\end{equation*}
where $\PV^p = \bigwedge^p \Xfrak(M)$ and differential
\begin{equation*}
  Q : v_1 \wedge \dots \wedge v_k \otimes 1 \longmapsto
  \sum_{i=1}^k (-1)^{i+1} v_1 \wedge \dots \wedge \hat{v}_i \wedge \dots \wedge v_k
  \otimes \bigl( p(v_i) - \drm S(v_i) \bigr)
\end{equation*}

\begin{exercise}
  Check that $\Hrm^0 (K^{\bullet}, Q) \cong \Oscr \bigl( \Graph (\drm S) \bigr)$, so
  \begin{equation*}
    \dCrit (S) \cong K^{\bullet} \otimes_{\Oscr(T^* M)} \Oscr (M) \cong \PV^{\bullet} (M)
  \end{equation*}
  and thus $\Oscr \bigl( \dCrit(S) \bigr) \simeq (\PV^\bullet, - \iota_{\drm S})$.
\end{exercise}

\begin{exercise}
   Show that $\Hrm^0 \Oscr( \dCrit ) \cong \Oscr (\Crit)$.
\end{exercise}

We can enhance $\Oscr \bigl( \dCrit (S) \bigr)$ to a sheaf on $M$. Following Grothendieck
\begin{equation*}
  \dCrit(S) = \bigl( M, \PV_M^\bullet, - \iota_{\drm S} \bigr)
\end{equation*}
is an example of a \textbf{dg manifold}.

\begin{definition}
  A dg manifold is a smooth manifold $M$ with a sheaf $\Oscr_M$ of \textbf{dg commutative algebras} (DGCAs) locally isomorphic to $\Oscr_M(U) \cong \bigwedge \Ecal (U)$, where $\Ecal$ are the smooth sections of $E \rightarrow M$.
\end{definition}

Ignoring the differential, we get a sheaf $\bigl(M, \PV_M^\bullet\bigr)$ on $M$ such that
\begin{equation*}
  \PV_M = \bigwedge \Xfrak_M \cong \Sym \Xfrak[1].
\end{equation*}
The underlying graded manifold of $\dCrit(S)$ is
\begin{equation*}
  T^*[-1]M = \bigl( M, \Sym \Xfrak[1] \bigr)
\end{equation*}
displaying the following properties:
\begin{enumerate}[i)]
  \item the graded manifold $T^*[-1]M$ is a $-1$-shifted symplectic graded manifold just as $T^*M$ is a $0$-shifted symplectic manifold;
  \item Induced from the $-1$-shifted symplectic structure we get a $1$-shifted Poisson bracket on $\Oscr \bigl( T^* [-1] M \bigr) = \PV(M)$ known as the \textbf{Schouten bracket}
  \begin{align*}
    \{ f, g \} &= 0, \\
    \{v, f \} &= v f, \\
    \{ v, w \} &= [v, w], \\
    \{u, v \wedge w \} &= \{u, v\} \wedge w + v \wedge \{ u, w \}
  \end{align*}
for $f, g \in \Oscr(M)$ and $u, v, w \in \PV^{-1}(M)$.
\end{enumerate}

\begin{exercise}
  Show that $-\iota_{\drm S} = \{ S, \cdot \}$.
\end{exercise}

\begin{definition}
  A $\Pbb_0$ algebra $\bigl( A, \drm, \{ \cdot, \cdot \} \bigr)$ is a DGCA $(A, \drm)$ equipped with a $-1$-shifted Poisson bracket $\{ \cdot, \cdot \}: A \otimes A \rightarrow A$ obeying:
  \begin{enumerate}[i)]
    \item \textbf{graded skew-symmetry:}
      \begin{equation*}
        \{x, y\} = - (-1)^{(|x|+1)(|y|+1)} \{y, x\};
      \end{equation*}
    \item \textbf{graded Poisson identity}:
      \begin{equation*}
        \{x, yz \} = \{x, y\} z + (-1)^{(|x|+1)|y|} y \{x, z\}
      \end{equation*}
      so $\{x, \cdot \}$ is a degree $|x|+1$ derivation;
    \item \textbf{graded Jacobi identity}:
      \begin{equation*}
        \{x, \{y, z\} \} = \{ \{x, y\}, z \} + (-1)^{(|x|+1)(|y|+1)} \{y, \{x, z\} \};
      \end{equation*}
    \item \textbf{compatibility with differential:}
      \begin{equation*}
        \drm \{x, y\} = \{ \drm x, y \} + (-1)^{|x|+1} \{x, \drm y \}.
      \end{equation*}
  \end{enumerate}
\end{definition}

\begin{exercise}
  Check that the Schouten bracket defines a $\Pbb_0$ algebra on $\Oscr\bigl(T^*[-1] M \bigr)$.
\end{exercise}

\chapter{Lecture 4}

Last time we constructed the derived critical locus of $S \colon M \to \Rbb$
\begin{equation*}
  \dCrit(S) = \bigl( T^*[-1] M, - \iota_{\drm S} \bigr)
\end{equation*}
where we understand $T^*[1] M = \bigl( M, \PV^\bullet \bigr)$ as the underlying manifold $M$ equipped with the sheaf defined by the assignment
\begin{equation*}
  \PV^\bullet \colon U \longrightarrow
  \Sym_{\Oscr(U)}^\bullet \bigl( \Tcal_M (U) [1] \bigr).
\end{equation*}

\begin{proposition}
    If $V$ is a locally-finite $-1$-shifted symplectic dg vector space then
    \begin{equation*}
      \Oscr(V) = \Sym(V^\vee)
    \end{equation*}
    is a $\Pbb$-algebra.
\end{proposition}
\begin{proof}
  The pairing $\langle \cdot, \cdot \rangle$ induces an isomorphism $V \cong V^\vee[-1]$ which we use to define a bracket
  \begin{equation*}
    \{\cdot, \cdot\} \colon \Sym^2(V^\vee) \longrightarrow \Rbb
  \end{equation*}
  which we extend as a derivation to $\Oscr(V)$.
\end{proof}

\section{Work Perturbatively}

Fix a solution to the equations of motion $p \in M$, and consider $V = T_p M$ instead of $M$.
Then we can expand $S$ as a polynomial (or formal power series).

For us, the space of fields $\Fcal$ will always be a sheaf of vector spaces on spacetime.
In our example, spacetime is a point $\text{pt}$ and $\Fcal = V$ for some finite-dimensional vector space.
Going forward we rewrite $T^* [-1]M \rightsquigarrow T^* [-1]V$.

\begin{remark}
  Be analogy to ungraded geometry $T^*V \cong V \oplus V^\vee$ we have that
  \begin{equation*}
    T^* [-1]V \cong V \oplus V^\vee [-1].
  \end{equation*}
  In infinite dimensions we consider the sheaves
  \begin{equation*}
    T^*[-1] \Fcal
    = \underbrace{\Fcal \oplus  \Fcal^{\vee}[-1]}_{\substack{
    \text{sheaf of BV fields}\\\text{without gauge symmetry}}}.
  \end{equation*}
\end{remark}

Because we are expanding around a critical point the action has takes the form
\begin{equation*}
  S(e) = \underbrace{\langle e, Q e \rangle}_{\substack{\text{Hessian} \\ \text{of }S}} + I(e), \qquad
  Q \colon \Fcal \longrightarrow \Fcal^{\vee}[-1].
\end{equation*}

Now let $M$ be a finite-dimensional manifold called spacetime, $\Fcal$ the space of naive fields, as a sheaf of vector spaces on $M$, and $\Ecal$ the sheaf of BV fields.
Our ultimate goal is to make sense of expressions of the form
\begin{equation*}
  \int_{\phi \in \Fcal(M)} \exp \Bigl( - \frac{S_{\text{naive}}}{\hbar} \Bigr) \DD \phi.
\end{equation*}
If $\Fcal = V$ and $S$ is quadratic then
\begin{align*}
  \int_V \exp \Bigl( - \frac{S(\phi)}{\hbar} \Bigr) \dd \phi
  &=
  \int_V \exp \Bigl( -\frac{\langle \phi, Q \phi \rangle}{\hbar} \Bigr) \dd \phi \\
  &=
  \Bigl( \frac{\pi}{\hbar} \Bigr)^{\frac{n}{2}} \det (Q)^{-\frac{1}{2}}
\end{align*}
and if $S(\phi) = \langle \phi, Q \phi \rangle + I(\phi)$ we incorporate the interaction terms by working perturbatively.

Even at finite dimensions, the case $\det (Q) = 0$ poses a bad problem when trying to apply the previous formula.
However, degenerate critical points are an unavoidable feature with gauge symmetry \footnote{Gauge transformations preserve the action $S^{\text{naive}}$ and the equations of motion.}
\begin{equation*}
  \underbrace{\Gcal \acts \Fcal.}_{\text{nonlinear action}}
\end{equation*}

\begin{example}
  Consider Yang-Mills with gauge group $G$ and trivial gauge bundle $M \times G \to M$.
  The space of fields is
  \begin{equation*}
    \Fcal(M) = \Omega^1 (M, \gfrak), \qquad \gfrak = \Lie (G)
  \end{equation*}
  and the group of gauge transformations
  \begin{equation*}
    \Gcal = \Aut (M \times G \longrightarrow M)
    \cong \Map (M, G)
  \end{equation*}
  where
  \begin{equation*}
    \Lie \Gcal \cong \Map(M, \gfrak) = \Omega^0 (M, \gfrak).
  \end{equation*}
  Instead of $\Gcal \acts \Fcal(M)$, we focus on the action of the Lie algebra of gauge transformations
  \begin{equation*}
    \Lie (\Gcal) \acts \Fcal(M).
  \end{equation*}
  Adopting the standard notation, we write that $c \in \Omega^0 (M, \gfrak)$ acts on $A \in \Omega^1 (M, \gfrak)$ by
  \begin{equation*}
    c \cdot A = \drm c + [c, A].
  \end{equation*}
\end{example}

\begin{exercise}
  Check that
  \begin{equation*}
    S_{\text{naive}}^{\text{YM}}
    = \int \langle F_A, F_A \rangle \drm \text{vol}
  \end{equation*}
  is invariant under infinitesimal gauge transformations.
\end{exercise}

\input{lecture5.tex}
\chapter{Lecture 6}

Last time we discussed \textbf{perturbative classical BV gauge theory}. We had
\begin{equation*}
  \Lcal \acts \Fcal
\end{equation*}
with $S_{\text{naive}} \in \Oscr_{\text{loc}}(\Fcal)$. The recipe is:
\begin{enumerate}[i)]
  \item take the \textbf{stacky quotient}
  \begin{equation*}
    \doublequotient{\Fcal}{\Lcal} = \Lcal[1] \oplus \Fcal
  \end{equation*}
  with a vector field $Q_{\text{CE}}$.
  The condition that $S_{\text{naive}}$ is gauge-invariant is equivalent to
  \begin{equation*}
    Q_{\text{CE}} S_{\text{naive}} = 0;
  \end{equation*}
  \item take the \textbf{derived critical locus}
  \begin{equation*}
    T^* [-1] \Bigl( \doublequotient{\Fcal}{\Lcal} \Bigr)
  \end{equation*}
  with differential $\{ S_{\text{naive}}, \cdot \}$.
  The underlying space is
  \begin{equation*}
    \Lcal[1] \oplus \Fcal \oplus \bigl( \Lcal[1] \oplus \Fcal \bigr)^{\vee}[-1]
    =\Lcal [1] \oplus \Fcal \oplus \Fcal^{\vee}[-1] \oplus \Lcal^{\vee}[-2].
  \end{equation*}
  \item obtain the \textbf{BV action} $S$ satisfying the CME
  \begin{equation*}
    \{ S, S \} = 0
  \end{equation*}
  and incorporate (somehow) $S_{\text{naive}}$ and $Q_{\text{CE}}$.
  This means that
  \begin{equation*}
    S_{\text{naive}} = S \big|_{\Fcal},
    \qquad \text{and} \qquad
    Q_{\text{CE}} = \{ S, \cdot \} \big|_{\Lcal[1] \oplus \Fcal}.
  \end{equation*}
\end{enumerate}

\begin{fact}
  The vector field $Q_{\text{CE}}$ is Hamiltonian, i.e.
  \begin{equation*}
    Q_{\text{CE}} = \{ S_{\text{CE}}, \cdot \}
  \end{equation*}
  with respect to the $-1$-shifted symplectic structure, for some $S_{\text{CE}}$.
  As such, we can define
  \begin{equation*}
    S = S_{\text{naive}} + S_{\text{CE}}.
  \end{equation*}
\end{fact}

For Yang-Mills on an oriented Riemannian $n$-manifold, with trivial bundle $M \times G \to M$ and $\gfrak = \Lie (G)$, $\Ecal$ looks like
\begin{equation*}
  \begin{tikzcd}
    \Omega^0(M, \gfrak) \arrow[r] &
    \Omega^1 (M, \gfrak) \arrow[r] &
    \Omega^{n-1} (M, \gfrak) \arrow[r] &
    \Omega^n (M, \gfrak)
  \end{tikzcd}
\end{equation*}
and the we write the action
\begin{equation*}
  S_{\text{naive}} (A) = \int_M \frac{1}{4} \langle F_A, F_A \rangle.
\end{equation*}
%TODO: Write Q_CE and compute S_CE.

We compute that
\begin{equation*}
  S_{\text{CE}}(c, A, A^*, c^*)
  = \int_M \langle \drm_A c, A^* \rangle + \frac{1}{2} \langle [c, c], c^* \rangle.
\end{equation*}
Choose bases $\{ T_a \}$ for $\gfrak$ and $\{ e_i \}$ for $V$.
An element of $gfrak[1] \oplus V$ can be written
\begin{equation*}
  X^a T_a [1] + v^i e_i, \qquad  X^a, v^i \in \Rbb.
\end{equation*}
Let $\xi^a$ be the linear coordinate corresponding to $T_a [1]$ and $x^i$ corresponding to $e_i$.
Note that $\xi^a$ has degree $+1$.

\begin{proposition}[Berezin, Leites]
  Consider
  \begin{equation*}
    \Der \Bigl( \Oscr(\Rbb^n) \otimes \bigwedge  W^{\vee} \Bigr)
  \end{equation*}
  where $\{ \theta_a \}$ is a basis of $W$, and $\theta^a$ denotes the respective dual basis elements. Then
  \begin{equation*}
    \Der \Bigl( \Oscr(\Rbb^n) \otimes \bigwedge W^{\vee} \Bigr)
    = \Bigl( \Oscr(\Rbb^n) \otimes \bigwedge W^{\vee} \Bigr) \biggl\{ \frac{\partial}{\partial x^i}, \frac{\partial}{\partial \theta^i} \biggr\}
  \end{equation*}
  where $\frac{\partial}{\partial \theta^i}$ is the degree $-1$ derivation such that
  \begin{equation*}
    \frac{\partial}{\partial \theta^i} \theta^{1} \dots \theta^{n}
    = (-1)^{i+1} \theta^1 \dots \hat{\theta}^k \dots \theta^n \delta_{i,k}.
   \end{equation*}
\end{proposition}

The idea of the computation is to determine the coefficients of the derivation $Q_{CE}$ by computing $Q_{\text{CE}} x^i$ and $Q_{\text{CE}} \xi^a$.
We obtain
\begin{equation*}
  Q_{\text{CE}} = \xi^a \rho_a^i \frac{\partial}{\partial x^i}
  - \underbrace{\frac{1}{2} f_{bc}^a \xi^b \xi^c \frac{\partial}{\partial \xi^a}}_{\frac{1}{2} [c, c]}
\end{equation*}
where $\rho: \gfrak \to \Xfrak(V)$ and $[T_a, T_b] = f_{a, b}^c T_c$.

\begin{definition}
  A \textbf{perturbative classical BV theory} consists of the data:
  \begin{enumerate}[i)]
    \item a graded vector bundle $E \to M$;
    \item a $-1$-shifted symplectic structure
    \begin{equation*}
      E \boxtimes E \longrightarrow \Dens M;
    \end{equation*}
    \item a local action functional $S \in \Oscr_{\text{loc}}(\Ecal)$ that is at least quadratic and satisfies the CME.
  \end{enumerate}
  Let $\Ecal$ be the topological vector space (TVS) of global smooth sections, and $\Ecal_c$ the TVS of compactly supported sections.
  Then $\Oscr(\Ecal)$ is the completion of the symmetric algebra on $\Ecal_c^{\vee}$.
  We define
  \begin{equation*}
    \Oscr(\Ecal)_{\text{loc}} (\Ecal) \subseteq \Oscr(\Ecal)
  \end{equation*}
  where $F \in \Oscr_{\text{loc}} (\Ecal)$ is a sum of terms of the form
  \begin{equation*}
    e \in \Ecal_c \longmapsto
    \int_M \Drm_1 e \dots \Drm_n e \Omega
  \end{equation*}
  for $\Drm_i \colon \Ecal \to \Crm^{\infty} (M)$ and $\Omega$ a density on $M$.
\end{definition}

\begin{proposition}
  For a classical BF theory $\Ecal$ we can write
  \begin{equation*}
    S = \int_M \langle e, Q e \rangle + I(e)
  \end{equation*}
  where $Q \colon \Ecal \to \Ecal$ is a differential operator of degree $+1$, squares to zero, and $I \in \Oscr_{\text{loc}}(\Ecal)$ is at least cubic and satisfies the QME
  \begin{equation*}
    Q I + \frac{1}{2} \{ I, I \} = 0.
  \end{equation*}
\end{proposition}


% ── Bibliography ──────────────────────────────────────────────────────
\sloppy
\printbibliography

\end{document}
