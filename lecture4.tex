\chapter{Lecture 4}

Last time we constructed the derived critical locus of $S \colon M \to \Rbb$
\begin{equation*}
  \dCrit(S) = \bigl( T^*[-1] M, - \iota_{\drm S} \bigr)
\end{equation*}
where we understand $T^*[1] M = \bigl( M, \PV^\bullet \bigr)$ as the underlying manifold $M$ equipped with the sheaf defined by the assignment
\begin{equation*}
  \PV^\bullet \colon U \longrightarrow
  \Sym_{\Oscr(U)}^\bullet \bigl( \Tcal_M (U) [1] \bigr).
\end{equation*}

\begin{proposition}
    If $V$ is a locally-finite $-1$-shifted symplectic dg vector space then
    \begin{equation*}
      \Oscr(V) = \Sym(V^\vee)
    \end{equation*}
    is a $\Pbb$-algebra.
\end{proposition}
\begin{proof}
  The pairing $\langle \cdot, \cdot \rangle$ induces an isomorphism $V \cong V^\vee[-1]$ which we use to define a bracket
  \begin{equation*}
    \{\cdot, \cdot\} \colon \Sym^2(V^\vee) \longrightarrow \Rbb
  \end{equation*}
  which we extend as a derivation to $\Oscr(V)$.
\end{proof}

\section{Work perturbatively}

Fix a solution to the equations of motion $p \in M$, and consider $V = T_p M$ instead of $M$.
Then we can expand $S$ as a polynomial (or formal power series).

For us, the space of fields $\Fcal$ will always be a sheaf of vector spaces on spacetime.
In our example, spacetime is a point $\text{pt}$ and $\Fcal = V$ for some finite-dimensional vector space.
Going forward we rewrite $T^* [-1]M \rightsquigarrow T^* [-1]V$.

\begin{remark}
  Be analogy to ungraded geometry $T^*V \cong V \oplus V^\vee$ we have that
  \begin{equation*}
    T^* [-1]V \cong V \oplus V^\vee [-1].
  \end{equation*}
  In infinite dimensions we consider the sheaves
  \begin{equation*}
    T^*[-1] \Fcal
    = \underbrace{\Fcal \oplus  \Fcal^{\vee}[-1]}_{\substack{
    \text{sheaf of BV fields}\\\text{without gauge symmetry}}}.
  \end{equation*}
\end{remark}

Because we are expanding around a critical point the action has takes the form
\begin{equation*}
  S(e) = \underbrace{\langle e, Q e \rangle}_{\substack{\text{Hessian} \\ \text{of }S}} + I(e), \qquad
  Q \colon \Fcal \longrightarrow \Fcal^{\vee}[-1].
\end{equation*}

Now let $M$ be a finite-dimensional manifold called spacetime, $\Fcal$ the space of naive fields, as a sheaf of vector spaces on $M$, and $\Ecal$ the sheaf of BV fields.
Our ultimate goal is to make sense of expressions of the form
\begin{equation*}
  \int_{\phi \in \Fcal(M)} \exp \Bigl( - \frac{S_{\text{naive}}}{\hbar} \Bigr) \DD \phi.
\end{equation*}
If $\Fcal = V$ and $S$ is quadratic then
\begin{align*}
  \int_V \exp \Bigl( - \frac{S(\phi)}{\hbar} \Bigr) \dd \phi
  &=
  \int_V \exp \Bigl( -\frac{\langle \phi, Q \phi \rangle}{\hbar} \Bigr) \dd \phi \\
  &=
  \Bigl( \frac{\pi}{\hbar} \Bigr)^{\frac{n}{2}} \det (Q)^{-\frac{1}{2}}
\end{align*}
and if $S(\phi) = \langle \phi, Q \phi \rangle + I(\phi)$ we incorporate the interaction terms by working perturbatively.

Even at finite dimensions, the case $\det (Q) = 0$ poses a bad problem when trying to apply the previous formula.
However, degenerate critical points are an unavoidable feature with gauge symmetry \footnote{Gauge transformations preserve the action $S^{\text{naive}}$ and the equations of motion.}
\begin{equation*}
  \underbrace{\Gcal \acts \Fcal.}_{\text{nonlinear action}}
\end{equation*}

\begin{example}
  Consider Yang-Mills with gauge group $G$ and trivial gauge bundle $M \times G \to M$.
  The space of fields is
  \begin{equation*}
    \Fcal(M) = \Omega^1 (M, \gfrak), \qquad \gfrak = \Lie (G)
  \end{equation*}
  and the group of gauge transformations
  \begin{equation*}
    \Gcal = \Aut (M \times G \longrightarrow M)
    \cong \Map (M, G)
  \end{equation*}
  where
  \begin{equation*}
    \Lie \Gcal \cong \Map(M, \gfrak) = \Omega^0 (M, \gfrak).
  \end{equation*}
  Instead of $\Gcal \acts \Fcal(M)$, we focus on the action of the Lie algebra of gauge transformations
  \begin{equation*}
    \Lie (\Gcal) \acts \Fcal(M).
  \end{equation*}
  Adopting the standard notation, we write that $c \in \Omega^0 (M, \gfrak)$ acts on $A \in \Omega^1 (M, \gfrak)$ by
  \begin{equation*}
    c \cdot A = \drm c + [c, A].
  \end{equation*}
\end{example}

\begin{exercise}
  Check that
  \begin{equation*}
    S_{\text{naive}}^{\text{YM}}
    = \int \langle F_A, F_A \rangle \dd \text{vol}
  \end{equation*}
  is invariant under infinitesimal gauge transformations.
\end{exercise}
