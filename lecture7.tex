\chapter{Lecture 7}

\section{Quantum BV in Finite Dimensions}

In this section we start out journey towards the quantization of BV theories.
Recall that out goal is to provide a \textbf{homological approach} to compute integrals of the form
\begin{equation*}
  \int_{\phi \in \Fcal} \euler^{- \frac{S}{\hbar}} \DD \phi.
\end{equation*}
We will introduce the quantum BV complex as a generalized \textbf{divergence complex}.
In finite dimensions this is an obscured version of the de Rham complex, where we have the usual homological approach to integration.

For this chapter we return to the finite-dimensional case, which corresponds to the case where spacetime is a point $M = \text{pt}$. Consider:
\begin{enumerate}[i)]
  \item a $n$-dimensional \textbf{graded} vector space $V = \sum_{n \in \Zbb} V_n$ of \textbf{fields};
  \item a $-1$-\textbf{shifted symplectic pairing} $\langle \cdot, \cdot \rangle \colon V \otimes V \to \Rbb$;
  \item an \textbf{action} $S \in \Oscr(V)$ such that
    \begin{equation*}
      \{ S, S \} =  0, \qquad \substack{\text{classical master} \\ \text{equation (CME)}}
    \end{equation*}
    where $\{ \cdot, \cdot \} \colon \Oscr(V) \otimes \Oscr(V) \rightarrow \Oscr(V)$ is the $+1$-\textbf{shifted Poisson bracket} induced by the symplectic pairing.
\end{enumerate}
Such a setup makes $\bigl( \Oscr(V), \{ \cdot, \cdot \} \bigr)$ into a cochain complex such that
\begin{equation*}
  \Hrm^0 \Oscr(V) \cong \Oscr( \Crit (S)).
\end{equation*}

Passing to the derived critical locus, we have seen that functions on $\dCrit(S)$ form a commutative dg algebra
\begin{equation*}
  \Oscr \bigl(\dCrit(S) \bigr) = \bigl( \PV^{\bullet} (V), -\iota_{\drm S} \bigr)
\end{equation*}
which we call the \textbf{classical BV complex}. Passing to the quantum version amounts to deforming this complex by changing the differential.

\section{Integration in Finite Dimensions}

Fixing a nonvanishing top form $\mu \in \Omega^{n}(V)$ defines a map
\begin{align*}
  \int_V \colon \Oscr(V)
  &\longrightarrow \Rbb \\
  f
  &\longmapsto \int_V f \mu
\end{align*}
where we make the necessary assumptions on $\Oscr$ such that $f$ is integrable with respect to $\mu$.
This map depends only on \textbf{cohomological data}. Explicitly, we have that
\begin{equation*}
  \int_M f \mu = \frac{\bigl[f \mu\bigr]}{\bigl[\mu\bigr]}
\end{equation*}
where $[\cdot]$ denotes the cohomology class in $\Hrm_{\text{dR}} (V)$.
Note that the denominator is just a normalizing factor.
If we pick $\mu \in \Omega^n(V)$ such that $\smallint_V \mu = 1$, computing the integral of $f$ with respect to $\mu$ boils down to computing a class in cohomology
\begin{equation*}
  \int_V f \mu = \bigl[f \mu\bigr].
\end{equation*}

The pairing by integration with $\mu$ is nondegenerate so it defines an isomorphism $\Oscr(V) \to \Omega^n(V)$ which we can extends to an isomorphism of complexes
\begin{align*}
  m_\mu \colon \PV^{-k} (V) 
  &\longrightarrow \Omega^{n - k} (V) \\
  X
  &\longmapsto \iota_X \mu.
\end{align*}

\begin{example}
  In coordinates, let $\mu = \drm x^1 \wedge \dots \wedge \drm x^n$ and $X = f \frac{\partial}{\partial x^{j_1}} \wedge \dots \wedge \frac{\partial}{\partial x^{j_k}}$. Then
  \begin{align*}
    m_{\mu} (X)
    &= f m_{\mu} \biggl( \frac{\partial}{\partial x^{j_1}} \wedge \dots \wedge \frac{\partial}{\partial x^{j_k}} \biggr) \\
    &= \sigma f \drm x^1 \dots \drm \hat{x}^{j_i} \dots \drm \hat{x}^{j_k} \dots \drm x^n
  \end{align*}
  where the terms $\drm \hat{x}^{j_i}$ are omitted, and $\sigma = \pm 1$ is such that
  \begin{equation*}
    \sigma \drm x^1 \dots \drm x^n
    = \drm x^{j_1} \dots \drm x^{j_k} \wedge \drm x^1 \dots \drm x^n.
  \end{equation*}
\end{example}

The \textbf{divergence operator} on $\PV^{\bullet}(V)$
\begin{equation*}
  \divergence_\mu = m_\mu^{-1} \drm m_\mu
\end{equation*}
is obtained by pulling back the de Rham differential on $\Omega^{\bullet}$ using $\mu$.
\begin{equation*}
  \begin{tikzcd}[sep=large]
    \dots 
    \arrow[r] &
    \PV^{-2}(V)
    \arrow[r, "\divergence_\mu"] \arrow[d, "m_\mu"] &
    \PV^{-1}(V)
    \arrow[r, "\divergence_\mu"] \arrow[d, "m_\mu"] &
    \Oscr(V)
    \arrow[d, "m_\mu"] \\
    \dots \arrow[r] &
    \Omega^{n-2}(V)
    \arrow[r, "\drm"] &
    \Omega^{n-1}(V)
    \arrow[r, "\drm"] &
    \Omega^{n}(V)
  \end{tikzcd}
\end{equation*}

Because it admits this definition in the finite-dimensional case, we say that the divergence complex is an \textit{obscured} version of the usual de Rham complex.
At this point, one might wonder: why do we not just use the de Rham complex to begin with?
The point is that, unlike the de Rham complex, the divergence complex generalizes to the infinite-dimensional case.

Recall how before we recovered the space of functions on the critical locus $\Crit(S)$ from $\PV^{\bullet}$ by passing to cohomology in degree $0$.
This crucial information is encoded in the de Rham complex in degree $n$, where top forms live.
In infinite dimensions, this data escapes as the de Rham complex ceases to be bounded above.
However, it still resides in degree $0$ in the divergence complex.
In this sense, the natural approach in quantum field theory is to generalize the divergence operator to the infinite-dimensional case.

\begin{example}
  Let $V = \Rbb$ and $\mu_\text{Leb}$ be the Lebesgue measure. A simple computation recovers the usual divergence operator
  \begin{align*}
    \divergence_\text{Leb} \biggl( f \frac{\partial}{\partial  x} \biggr)
    &= m_\mu^{-1} \drm m_\mu \biggl( f \frac{\partial}{\partial x} \biggr) \\
    &= m_\mu^{-1} \bigl( \drm f\bigr) \\
    &= m_\mu^{-1} \biggl( \frac{\partial f}{\partial x} \mu \biggr) \\
    &= \frac{\partial f}{\partial x}.
  \end{align*}
  If we write the generator of the vector fields as $\xi = \frac{\partial}{\partial x}$ then
  \begin{equation*}
    \divergence_\text{Leb} = \frac{\partial}{\partial x} \frac{\partial}{\partial \xi} = \laplace_{\text{BV}}
  \end{equation*}
  takes the form of the usual BV Laplacian.
  It is straightforward to generalize to the $n$-dimensional case
  \begin{equation*}
    \divergence_{\text{Leb}} = \laplace_{\text{BV}}
    = \sum_{i = 1} \frac{\partial}{\partial x^i} \frac{\partial}{\partial \xi^i}.
  \end{equation*}
\end{example}

\begin{example}
  Consider again $V = \Rbb$ but let $\mu_S$ be a Gaussian measure of the form
  \begin{equation*}
    \mu_S = \euler^{-\frac{S}{\hbar}} \mu_\text{Leb}.
  \end{equation*}
  In this case, we see that
  \begin{align*}
    \divergence_S 
    &= m_{\mu_S}^{-1} \drm m_{\mu_S} \biggl( f \frac{\partial}{\partial x} \biggr) \\
    &= m_{\mu_S}^{-1} \drm \Bigl( f \euler^{-\frac{S}{\hbar}} \Bigr) \\
    &= m_{\mu_S}^{-1} \biggl( \frac{\partial f}{\partial  x} \euler^{-\frac{S}{\hbar}} \mu_\text{Leb}
    - \frac{1}{\hbar} \frac{\partial S}{\partial x} f \euler^{-\frac{S}{\hbar}} \mu_\text{Leb} \biggr) \\
    &= \frac{\partial f}{\partial  x} - \frac{1}{\hbar} \frac{\partial S}{\partial  x} f.
  \end{align*}
  from which we conclude that
  \begin{equation*}
    \divergence_S = - \frac{1}{\hbar} \iota_{\drm S} + \laplace_\text{BV}.
  \end{equation*}
  If $\hbar \neq 0$ (and is not formal) we can multiply by $\hbar$ to obtain a differential
\begin{equation*}
    \hbar \divergence_S = - \iota_{\drm S} + \hbar \laplace_\text{BV}
  \end{equation*}
  on $\PV^{\bullet}(V)$ that we recognize as a deformation of the classical BV differential given by contracting with $\drm S$. Alternatively, we can also write the divergence operator in terms of the Schouten bracket
  %TODO: reference the Schouten bracket form previous lecture
  \begin{equation*}
    \divergence_S = \{ S, \cdot \} + \hbar \laplace_{\text{BV}}.
  \end{equation*}
\end{example}
