\chapter{Lecture 1}

\section{History}

A brief history of the development of the Batallin-Vilkovisky (BV) formalism:

\begin{center}
  \begin{tabular}{ccccc}
    Date & People & What & Why & Techniques \\
    \midrule
    1969 &
    \makecell{ Faddeev \\ and Popov } &
    \makecell{ Gauge fixing \\ (adding ghosts) } &
    \makecell{ Quantize \\ Yang-Mills } &
    \makecell{ Berezinian \\ integration } \\
    \addlinespace
    1973 & 
    \makecell{ 't Hooft and \\ Veltman } &
    \makecell{ Quantized \\ Yang-Mills } &
    \makecell{ Quantize \\ Yang-Mills } &
    \makecell{ Feynman \\ diagrams } \\
    \addlinespace
    1975 &
    \makecell{ Becchi, Rouet, \\ Stora, Tyutin \\ (BRST) } &
    \makecell{ Cohomological \\ theory to quantize \\ Yang-Mills } &
    \makecell{ Understanding \\ 't Hooft \\ and Veltman } &
    \makecell{ Derived invariants \\ (Lie algebra \\ cohomology) } \\
    \addlinespace
    1981 &
    \makecell{ Batallin and \\ Vilkovisky \\ (BV) } &
    \makecell{ Quantize systems \\ with complicated \\ gauge symmetries } &
    Supergravity &
    \makecell{ Derived \\ intersections \\ (Koszul complexes) } \\
    \addlinespace
    1992 & Henneaux &
    \makecell{ Quantize \\ Yang-Mills \\ using BV } &
    \makecell{ Analyze \\ Yang-Mills \\ using BV } &
    \makecell{ Derived \\ intersections} \\
    \addlinespace
    2007 & Costello &
    \makecell{ Combine BV \\ with effective \\ field theory } &
    \makecell{ Make BV \\ quantization \\ rigorous } &
    \makecell{ Derived everything, \\ analysis, and \\ homotopy theory }
  \end{tabular}
\end{center}

% ╭──────────────────────────────────────────────────────────╮
% │                        References                        │
% ╰──────────────────────────────────────────────────────────╯
\section{References}

The main references will be:

\begin{itemize}
  \item Costello - Renormalization and Effective Field Theory \cite{CosRenormalization11};
  \item Elliot, Williams, Yoo - Asymptotic Freedom in the BV Formalism \cite{EWYAsymptotic18};
  \item Gwilliam - Factorization algebras and free field theories \cite{GwiFactorization}.
\end{itemize}

% ╭──────────────────────────────────────────────────────────╮
% │                Roadmap to BV quantization                │
% ╰──────────────────────────────────────────────────────────╯
\section{Roadmap to BV quantization}

\begin{figure}
  \includestandalone{roadmap_to_BV}
  \caption{Roadmap to BV quantization.}
\end{figure}

The space of fields $\Ecal^\bullet$ is a cochain complex
\begin{equation*}
  \begin{tikzcd}
    \dots \arrow[r] &
    \Ecal^{-1} \arrow[r, "Q"] &
    \Ecal^0 \arrow[r, "Q"] &
    \Ecal^1 \arrow[r, "Q"] &
    \Ecal^2 \arrow[r] &
    \dots
  \end{tikzcd}
\end{equation*}
equipped with a differential $Q$ such that $Q^2 = 0$. Moreover, $\Ecal$ admits a $-1$-shifted symplectic structure, that is, there exists a nondegenerate pairing of degree $-1$
\begin{equation*}
  \langle \cdot , \cdot \rangle \colon
  \Ecal \otimes \Ecal \longrightarrow \Rbb [-1]
\end{equation*}
such that $\langle x, y \rangle = -(-1)^{(|x|+1)(|y|+1)} \langle y, x \rangle$.
This structure defines a $+1$-shifted Poisson bracket
\begin{equation*}
  \{ \cdot, \cdot \} \colon
  \Oscr( \Ecal ) \otimes \Oscr( \Ecal ) \longrightarrow \Oscr( \Ecal )
\end{equation*}
where $\Oscr ( \Ecal ) \cong \mathrm{Sym}^\bullet (\Ecal^\vee)$ is the (graded) commutative algebra of polynomial functions on the dual complex $\Ecal^\vee$.
Consider a function $S \in \Oscr (\Ecal)$ obeying the \textbf{classical master equation} (CME)
\begin{equation*}
  \{ S, S \} = 0.
\end{equation*}
The data $(\Ecal^\bullet, \langle \cdot, \cdot \rangle, S)$ defines a \textbf{classical BV theory}.
The CME implies that $\{ S, \cdot \}$ is a differential making $(\Oscr ( \Ecal ), \{S, \cdot \} )$ into a cochain complex such that
\begin{equation*}
  \Hrm^0 \Oscr ( \Ecal ) \cong
  \Oscr( \mathrm{Crit} (S) ),
\end{equation*}
where $\mathrm{Crit} (S)$ denotes the critical locus of $S$.
We will restrict to $S$ of the form
\begin{equation*}
S(e) = \underbrace{\langle e, Qe \rangle}_{\substack{ \text{free part} \\ \text{(kinetic +} \\ \text{mass terms)} }}
+ \underbrace{I(e)}_{\substack{ \text{interaction} \\ \text{part (cubic} \\ \text{or higher)} }}.
\end{equation*}

\begin{example}
  Why are the cubic and higher order terms called interaction terms?
For electromagnetism on a manifold $M$ we have a space of fields
  $\Fcal = \Omega^1(M) \oplus \Omega^0(M, S)$ in degree $0$. Let $F = \drm A$ and define
  \begin{equation*}
    S(A, \psi) = \int_M
    \underbrace{F \wedge \hodge F
    + \langle \psi, \slashed{\drm} \psi \rangle \dd \mathrm{vol}}_{\text{quadratic terms}}
    + \underbrace{\langle \psi, \slashed{A} \psi \rangle \dd \mathrm{vol}}_{\text{interaction terms}}.
  \end{equation*}
  Computing the Euler-Lagrange equations we obtain the system of differential equations
  \begin{align*}
    \hodge \drm \hodge F &= \bar{\psi} \gamma^\mu \psi \dd x_\mu \\
    \slashed{d}_A \psi &= 0
  \end{align*}
  which is coupled because of the interaction term.
\end{example}

% ╭──────────────────────────────────────────────────────────╮
% │             Quantization in the BV formalism             │
% ╰──────────────────────────────────────────────────────────╯
\section{Quantization in the BV formalism}

The slogan of quantization in the BV formalism is to \textit{deform the differential}.
In the perturbative context we work in formal power series in $\hbar$, for example, over the ring $\Rbb \formalpow{\hbar}$.
Quantization results in a cochain complex
$(\Oscr(\Ecal)\formalpow{\hbar}, \{S^q, \cdot\} + \hbar \laplace)$, where $\laplace$ is called the BV Laplacian, and the function
$S^q \in \Oscr(\Ecal)[[\hbar]]$ satisfies the \textbf{quantum master equation} (QME)
\begin{equation*}
  (\{S^q, \cdot \} + \hbar \laplace)^2 = 0.
\end{equation*}

\begin{example}
  In finite dimensions, consider the BV fields
  \begin{equation*}
    \Ecal^\bullet =
    (\begin{tikzcd}[sep=small]
      \Rbb \arrow[r] & \Rbb
    \end{tikzcd}).
  \end{equation*}
  concentrated in degree $0$ and $1$.
  Then
  \begin{equation*}
    \Oscr(\Ecal) \cong \Rbb \bigl[ x^1, \dots, x^n, \xi^1, \dots, \xi^n \bigr]
  \end{equation*}
  and the BV Laplacian takes the form
  \begin{equation*}
    \laplace = \sum_{\mu = 1}^n \frac{\partial}{\partial \xi^\mu} \frac{\partial}{\partial x^\mu}.
  \end{equation*}
  Note that $\laplace$ is a differential operator of degree $1$ such that $\laplace^2 = 0$.
\end{example}
The quantum action is a function of the form
\begin{equation*}
  S^q(e) = \langle e, Qe \rangle
  + I^q(e)
\end{equation*}
where $I^q \in \Oscr(\Ecal) [[\hbar]]$ is cubic mod $\hbar$ and satisfies the QME
\begin{equation*}
  Q I^q + \frac{1}{2} \{I^q, I^q\} + \hbar \laplace I^q = 0
\end{equation*}
which resembles the \textbf{Maurer-Cartan} (MC) equation. In infinite dimensions, some problems arise:

\begin{enumerate}[i)]
  \item there may be no solution to this equation. In this case we say that quantization is obstructed (there is an anomaly);
  \item the QME in infinite dimensions is ill-defined. Some functional analysis is needed to make sense of this problem.
\end{enumerate}
